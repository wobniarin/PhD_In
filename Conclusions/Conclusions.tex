
\chapter{Conclusions}
\label{conclus}
    %Introduction to the general conclusions  
    The increase of the electricity consumption in certain time periods, combined with the increase of DERs in distribution networks has made 
    This thesis gathers several studies related to flexibility services, and each chapter outlines specific conclusions based on the results. Despite, the present chapter exposes the main conclusions of the thesis, summarizes the main contributions and draw further research required for the resolution of remaining questions. 

\section{General conclusions}
	
    
     
\section{Contributions}
The main contributions are listed below: 
\begin{itemize}
\item Chapter XXX: 
	\begin{itemize}
		\item XXXX
		\item XXXX
	\end{itemize}
\item Chapter XXX: 
	\begin{itemize}
		\item XXXX
		\item XXXX
	\end{itemize}
\item Chapter XXX: 
	\begin{itemize}
		\item XXXX
		\item XXXX
	\end{itemize}
\end{itemize}
	


\section{Future work}
The presence of flexibility in the distribution network is already a reality, and at the moment it can be considered a trend. Therefore, further research to improve the state-of-the-art forecasting techniques and optimization models for the activation of this flexibility for DSOs is required, and will be continuously developed in the upcoming years. 
Considering the research developed in each of the chapters of this manuscript, the potential research areas based on the accomplished achievements of the present thesis are described below: 

\begin{itemize}
\item Chapter XXX: 
	\begin{itemize}
		\item XXXX
		\item XXXX
	\end{itemize}
\item Chapter XXX: 
	\begin{itemize}
		\item XXXX
		\item XXXX
	\end{itemize}
\item Chapter XXX: 
	\begin{itemize}
		\item XXXX
		\item XXXX
	\end{itemize}
\end{itemize}
