%%%%%%%%%%%%%%%%%%%%%%%%%%%%%%%%%%%%%%%%%%%%%%%%%%%%%%%%%%%%%%%%%%%%%%%%%%%%%%%%%%%%%%%%%%%%%%%%%%%
% Resum de la tesi																																								%
%%%%%%%%%%%%%%%%%%%%%%%%%%%%%%%%%%%%%%%%%%%%%%%%%%%%%%%%%%%%%%%%%%%%%%%%%%%%%%%%%%%%%%%%%%%%%%%%%%%
\begin{otherlanguage}{English}
\renewcommand{\abstractname}{Abstract}
\begin{abstract}	

On the way towards a low carbon electricity system, flexibility has become one of the main sources for achieving it. Flexibility can be understood as the ability of a power system to cope with the variability and uncertainty of demand and supply. Both the generation-side and the demand-side can provide it. This research is focused on the role of the demand-side flexibility for providing a service to the distribution system operator, who manages the medium and low-voltage network. By activating this flexibility from the demand-side to the distribution network operator, the latter can avoid or mitigate congestions in the network and prevent grid reinforcement. 

This thesis starts with analyzing the current state of the art in the field of local electricity markets, setting the baseline for flexibility products in the power systems. As a result of the previous analysis, the definition of flexibility is developed more specifically, considering the flexible assets to be controlled, the final client using this flexibility and the time horizon for this flexibility provision. 

Following the previous step, an aggregated flexibility forecast model is developed, considering a flexibility portfolio based on different flexible assets such as electric vehicles, water boilers, and electric space heaters. The signal is then modeled under a system-oriented approach for providing a service to the distribution network operator under the operation timeline on a day-ahead basis. The flexibility required by the distribution network operator is then calculated through an optimization problem, considering the flexibility activation costs and the network power flow constraints. 

Finally, since this scenario aims to lower the environmental impacts of the power system, its sustainability is assessed with the life-cycle assessment, considering the entire life cycle and evaluating it in terms of greenhouse gas emissions. Using this approach enhances the analysis of the potential role of flexibility in the power system, quantifying whether, in all cases, there is a reduction of emissions when shifting the consumption from peak hours to non-peak hours. 


\end{abstract}
\end{otherlanguage}

\begin{otherlanguage}{Catalan}
\renewcommand{\abstractname}{Resum}
\begin{abstract}
En el cam\'{i} cap a un sistema el\`{e}ctric amb baixes emissions de carboni, la flexibilitat s'ha convertit en una de les principals fonts per aconseguir-ho. La flexibilitat es pot entendre com la capacitat d'un sistema de reaccionar davant la variabilitat i la incertesa provocades per la demanda i la generaci\'{o}. Tant la part de la generaci\'{o} com el costat de la demanda tenen actius per poder proporcionar-ho. La recerca presentada en aquest manuscrit est\`{a} enfocada el paper de la flexibilitat oferta per la demanda, per a proporcionar un servei a l'operador del sistema de distribuci\'{o}, que gestiona la xarxa de mitja i baixa tensi\'{o}. Gr\`{a}cies a l'activaci\'{o} de la flexibilitat del costat de la demanda, l'operador de les xarxes de distribuci\'{o} pot evitar o mitigar la congesti\'{o} de la xarxa i evitar-ne les inversions per a refor\c{c}ar-la.

Aquesta tesi comen\c{c}a amb l'an\`{a}lisi de l'estat de l'art en el camp dels mercats d'electricitat locals, establint-ne la l\'{i}nia base per a la definici\'{o} dels productes de flexibilitat en els sistemes el\`{e}ctrics. Com a resultat de l'estudi anterior, la definici\'{o} de flexibilitat es desenvolupa m\'{e}s espec\'{i}ficament, considerant els actius flexibles que han de controlar-se, el client final que utilitza aquesta flexibilitat i l'horitz\'{o} temporal per a aquesta disposici\'{o} de flexibilitat.

A continuaci\'{o} es desenvolupa un model de previsi\'{o} de flexibilitat agregada, considerant una cartera de flexibilitat basada en diferents actius flexibles, com ara vehicles el\`{e}ctrics, calderes d'aigua, i escalfadors el\`{e}ctrics. El senyal es modela sota un enfocament orientat al sistema per proporcionar un servei a l'operador de la xarxa de distribuci\'{o}, per un horitz\'{o} temporal corresponent a l'operaci\'{o} de la xarxa de mitja tensi\'{o}. M\'{e}s endavant, la flexibilitat requerida per l'operador de la xarxa de distribuci\'{o} es calcula a trav\'{e}s d'un problema d'optimitzaci\'{o}, tenint en compte els costos d'activaci\'{o} de la flexibilitat i les restriccions de flux de pot\`{e}ncia de la xarxa de distribuci\'{o}.

Finalment, at\`{e}s que aquest escenari pret\'{e}n reduir l'impacte mediambiental del sistema el\`{e}ctric, la seva sostenibilitat s'avalua considerant tot el cicle de vida de totes les tecnologies que hi participen, i avaluant-lo en termes d'emissions de gasos d'efecte d'hivernacle. L'\'{u}s d'aquest enfocament millora l'an\`{a}lisi del potencial paper de la flexibilitat en el sistema el\`{e}ctric, quantificant si, en tots els casos, hi ha una reducci\'{o} de les emissions traslladant el consum de les hores punta a hores vall.
\end{abstract}
\end{otherlanguage}

%In this case, the local markets can provide two main services: energy and flexibility, and two approaches are considered: the peer-to-platform or centralised approach, and the peer-to-peer approach. This literature review provides a clear definition for all the agents involved, emphasizing the role of the aggregator, being the key agent in this PhD research as the agent managing the flexibility from the demand-side and providing it to the distribution network operator. 
%As a result of the previous analysis, the definition of flexibility is developed in a more specific way, considering the flexible assets to be controlled, the final client using this flexibility as well as the time horizon for this flexibility provision. This exercise leads to the definition of the final flexibility signal considered in this research to be modelled and estimated. In this PhD research, the flexibility is collected by a set of end-users or prosumers, with a varied portfolio of flexible assets covering from electric vehicles, water boilers and electric space heaters. The signal is then modelled under a system-oriented approach, for providing a service to the distribution network operator under the operation timeline, in a day-ahead basis. 
%This approach is then implemented under a hierarchical framework, for forecasting the available flexibility in a two-level hierarchy. The first level determines whether there is flexibility available in the portfolio, whereas the second level determine its value under the cases where there is availability of the product. This is done using a probabilistic forecast method by means of recursive maximum likelihood kernel density estimation. 


% \textbf{Storytelling PhD}
% \begin{itemize}
% \item What is the current situation?
% \begin{itemize}
%	\item  Energy Transition - Clean Energy Package
%	\item Increase on electricity consumption
%	\item decomissioning of nuclear power plants
%	\item Importance on society awareness on sustainability, as well as prices in some countries for access to electricity - What is our current carbon footprint?
%  \end{itemize}
%  
% \item What are the problems that we identify?
%	\begin{itemize}
%	\item some of the solutions for the energy transition can lead to congestion in distribution networks, due to the placement of DERs in MV and LV networks. 
%	\item Are we sure that DERs are completely sustainable? Are we really sure that smart grids are completely sustainable, in each and every country? 
%	\end{itemize}	 
% 
% \item Possible solutions
% \begin{itemize}
% \item Enhance the development of carbon-free technologies by implementing sustainable solutions - Assessment of the environmental impact of electricity production by means of LCA to help in the development of energy policies and planning, as well as including LCA in end-user optimization for achieving lower carbon footprints. 
% \item Deployment of demand-side flexibility services by means of end-users flexible assets. 
% \end{itemize}
% \item Thesis contributions
% \begin{itemize}
% 	\item Assessment tool for evaluating the environmental impact of electricity markets using hourly LCA, by means of Life Cycle Assessment
% 	\item Definition of Flexibility: Characterization, modeling, implementation and evaluation
% 	\item Possible use cases of demand-side flexibility: Flexibility services for DSOs. 
% \end{itemize}
% \end{itemize}
