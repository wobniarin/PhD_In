%\chapter{Aggregated flexibility forecast within aggregators flexibility portfolios}
\chapter{Demand-side flexibility forecast for aggregators}
\label{ChapterAggFlexForecast}
\chaptermark{Aggregated Flexibility Forecast}

\section{Objectives and contributions}
% Que es la flexibilitat i com ens podem beneficiar d'ella
Flexibility in smart grids has become a key element to enhance the integration of renewable energy sources  that  are  variable and  with some  natural  uncertainty  associated  to  them \cite{Ulbig2015, Goutte2019}. Furthermore, the increase in electricity consumption in specific time periods can lead to network operation problems such as congestions \cite{CEDEC2018, LaurA.Nieto-MartinJ.BunnD.Vicente-Pastor2019}. One way of activating flexibility is by Demand-Side Management (DSM), incentivizing the consumption through electricity price signals, allowing a paradigm shift where consumption follows generation partially \cite{Strbac2008}. Another way is by aggregators providing flexibility services to the Distribution System Operator (DSO) \cite{MUNNE-COLLADO2019}, the Balance Responsible Party (BRP) or retailers under a Local Flexibility Market (LFM) \cite{Olivella-Rosell2018, Heinrich2020}. Thus, aggregators must directly control the end-user's assets to increase or decrease the electricity consumption at specific time periods. For this purpose, aggregators should know the potential flexibility out of the total load. 
The contributions of this chapter are (i) the development of a framework based on hierarchical modeling to characterize and predict the aggregated flexibility within a flexibility portfolio; (ii) a probabilistic forecast formulation of the aggregated flexibility based on Online Learning, using Kernel Density Estimation and Recursive Maximum Likelihood; (iii) a flexibility forecast approach that does not require network topology information; and (iv) a flexibility estimation that is applicable to different flexible assets, and does not require specific information of them. 

This chapter aims to provide a probabilistic tool for estimating the available flexibility of a set of flexible assets managed by an aggregator. Hence, the interaction studied is the one according to objective $(iii)$ of the PhD thesis, outlined in Figure \ref{fig:chapter_obj_iii}. The organization of the chapters is the following. Section \ref{Sect:ProblemStatement} introduces the definition of flexibility and the modeling approach. Section \ref{Sect:FlexModeling} describes the two-level hierarchy chosen for the flexibility modeling and the mathematical formulation. Section \ref{Sect:CaseStudy} presents a case study of the aggregated flexibility forecast under a real dataset, while Section \ref{sect:Results} discusses the obtained results under the case study. Finally, Section \ref{Sect:Conclusions} concludes on the results.

\begin{figure}[h]
	\centering
	\includegraphics[width=0.7\columnwidth ]{ChapterAggFlexForecast/Figures/phd_intro_iii.pdf}
	   %\vspace*{-8cm}
		\caption{Chapter objective based on the PhD scope}
	\label{fig:chapter_obj_iii}  
\end{figure}

\subsection{Literature review}
% Que s'ha fet fins ara quan parlem de flexibility forecast
Several works in the literature have investigated different approaches for providing flexibility in the electrical network to DSOs, BRPs, or retailers, highlighting the feasibility and the advantages of these services \cite{Huber2020, Bodal2017, Gorria2013, Pinto2017,Lucas2019, Yue2020}. There are mainly three ways of carrying out this flexibility forecast, that is by means of individual forecast at each household through a Home Energy Management System (HEMS) \cite{Pinto2017}, individual forecast by asset type \cite{Huber2020}, or by providing an aggregated forecast of the portfolios' flexibility \cite{Bodal2017}. Flexibility can be forecast aggregately by modeling the aggregated electricity demand of a group of domestic users signed up to an incentive-based DSM program \cite{Gorria2013}. Another approach, and certainly one of the most common, is Non-Intrusive Load Metering (NILM) to obtain the flexibility value from residential users, as implemented in  \cite{Lucas2019, Yue2020}. %Nonetheless, this research focuses on distribution networks, from MV to LV feeders, under the control and operation of DSOs. Besides, BRPs or retailers may also require flexibility to reduce the imbalances between generation and demand in their portfolios, and hence, penalties.
%Limitacions del que s'ha fet fins ara   
There is still room for improvement in terms of flexibility modeling and forecasting due to limitations: (i) The existing flexibility forecast models assume known network topology and all the information regarding the flexible assets at each of the households \cite{Pinto2017, Gorria2013}. However, the current regulation states that aggregators and DSOs must be different entities \cite{Guldbaek2017, BEUC2018, EuropeanParliament2019}, complicating the implementation of such services due to the lack of network-related data sharing. (ii) Aggregators might not have access to asset-specific data due to data storage, information availability limitations, or data privacy and protection such as the General Data Protection Regulation (GDPR) \cite{GDPR1, GDPR2}. (iii) Forecasting at each individual household and then aggregating can lead to computation times longer than the operation times required for providing flexibility services \cite{Olivella2020}. These differences on levels of information, business model interests as well as conflicting objectives among DSOs and aggregators are also pointed out by \cite{Heinrich2020}. The reviewed literature shows a research gap on how flexibility can be defined and estimated, avoiding the use of asset-specific data and providing probabilistic forecast to tackle the uncertainty associated with demand-side flexibility.

Kernel Density Estimation (KDE) methods are commonly used for obtaining predictive distributions of a specific signal, being commonly parametrized with a mean-variance model when data do not follow a parametric distribution. This approach also allows to be implemented online, considering the evolution of the data density function as soon as new data points enter the model, allowing this approach to be used for probabilistic forecast. KDE for renewable energy forecasting has been implemented in literature \cite{Shi2020}, being mainly applied to wind energy forecast \cite{Pinson2009}. In \cite{Shi2020}, a conditional KDE is implemented to forecast solar and wind energy generation, using an adaptive bandwidth with the aim of minimizing the associated error. Recent research has shown an increase of the use of this approach on load forecasting \cite{Wang2019, Haben2016}. In \cite{Wang2019}, KDE is used to calculate the medium term probabilistic load consumption forecast, for energy planning. Until now, KDE has only been focused on a single asset-type data and mainly for planning purposes, being the implementation of KDE for aggregated flexibility forecast for portfolio operation not considered yet.

% què presentem nosaltres?? 
This chapter presents a methodology for estimating the flexibility by means of Online KDE, employing Gaussian kernels, which are parameterized with a mean-variance model. Accordingly, the relevant parameters of the kernels are tracked with a Recursive Maximum Likelihood estimation method. Recursivity and on-line learning approaches outlined here allow the time-adaptivity of the model, and potential application to real test cases. This methodology provides the aggregator with a tool to estimate the flexibility availability probability, as well as its conditional value, in a short period of time, for operation purposes, without the need of computing HEMS optimization algorithms for each household. Furthermore, in this approach no particular forecasting models for each asset type are needed, since the estimation is done in an aggregated level, only using metering and submetering data, assuming that the flexibility signal is known. An additional advantage is that asset-specific data such as driving patterns or battery state of charge, among others, are not needed, which are usually not available. This is because the presented methodology is general and asset-independent. Hence, this approach is useful for decision-making objectives in an aggregated approach as a first stage of the flexibility provision.


\section{Aggregated flexibility estimation}\label{Sect:ProblemStatement}

\subsection{Problem statement} \label{sec:problemstatement}
By considering all the previous definitions in Chapter 3, and the main objective of this study, flexibility is defined and formulated as (i) aggregated, by jointly considering a group or a portfolio of users represented by an aggregator, with no available information neither in terms of the electrical network layout nor the location of the flexible assets; (ii) consumption approach, since flexibility is modeled considering only those flexible sources that consume energy, being prosumption \cite{Ritzer2010} out of the scope at this stage; (iii) short-term horizon, since flexibility will be forecast in a day-ahead basis, in time periods that may range from 15 minutes to 1 hour; and lastly (iv) system-oriented, being the output of this algorithm the energy value, defined as positive and in energy units [kWh], for operation and short-term decision-making purposes for DSOs and BRPs, with no associated price or cost. 

\subsection{Approaches for Flexibility Aggregation}
With the aim of characterizing and modeling flexibility based on an aggregated portfolio with different sources of flexible consumption, bottom-up and top-down approaches are used. Instead of modeling and forecasting each type of flexibility source, the aggregated flexibility value is predicted. Figure \ref{fig:bottom_up} shows the bottom-up approach used to obtain the initial dataset to model the aggregated flexibility value. By means of this approach, the signal obtained will be later used to characterize the flexibility signal and predict its value. In this model, three different sources of flexibility are considered based on submetering data: Electric Water Boilers (EWB), Space Heaters (SH), and Electric Vehicles (EV). The aggregated flexibility value can be obtained by adding them up by type and user, which will be the input data for the flexibility characterization and modeling.

\begin{figure}[htbp]
\centerline{\includegraphics[width=0.9\columnwidth]{ChapterAggFlexForecast/Figures/BUAP.pdf}}
\caption{Bottom-up approach for flexibility modeling}
\label{fig:bottom_up}
\end{figure}

The second stage considers the aggregated flexibility from the bottom-up approach as input data. Then, the modeling of the flexibility signal is defined by employing a two-level hierarchical model and top-down approach, as shown in Figure \ref{fig:top_down}. The first level of the hierarchy characterizes the flexibility signal. In this context, the signal is transformed and modeled as a Bernoulli distribution, characterizing the flexibility signal into two different values based on a chosen threshold; flexibility available (1) and flexibility not available (0). By doing that, one can first know whether there is flexibility available in the portfolio before quantifying the available amount under the second level of the hierarchy. 

\begin{figure}[htbp]
\centerline{\includegraphics[width=0.7\columnwidth]{ChapterAggFlexForecast/Figures/TPAP.pdf}}
\caption{Top-down approach for flexibility characterization and modeling.}
\label{fig:top_down}
\end{figure}

\section{Flexibility modeling} \label{Sect:FlexModeling}
\subsection{Dealing with time series data}
The approach to forecasting depends on the time horizon, the factors and related inputs that determine the current outcome, types of visible and "not visible" data patterns, and many more. There are several approaches for forecasting time series data, from the most basic used for developing a benchmark, also know as na\"{i}ve methods, to the more complex ones such as probabilistic forecasts, neural networks, or reinforcement learning \cite{Hyndman2021}. This chapter covers the five main steps in a forecasting project, as follows:
 
\begin{enumerate}
\item \textbf{Problem definition}: Definition of the variable to be predicted, as well as the integration of this forecast model to the organisation and the related models that might interact with it. 
\item \textbf{Data and information collection}: According to \cite{Hyndman2021}, there are at least two types of information required: (a) being the statistical data, and (b) the accumulated expertise of people who collect the data and use these forecasts. On top of that, even though larger amounts of data are currently being stored, there are still some difficulties to collect enough historical data to be able to fit a good statistical model. 
\item \textbf{Exploratory Data Analysis (EDA)}: With the main objective to find consistent patterns such as seasonality and trends, as well as outliers and missing values. 
\item \textbf{Choosing and fitting models}: This step is directly related to the amount of historical data available and other explanatory variables, since they will affect the choice of the model to use. Most likely, the definition of a benchmark will help to evaluate in a later stage the performance of the more complex model. It is important to remember here that each model is itself an artificial construct based on a set of assumptions. and usually invovles one or more parameters which must be estimated using the available and known historical data  \cite{Hyndman2021}. 
\item \textbf{Model evaluation and utilization}: Evaluating the performance of the model once the data for the forecast period have become available.
\end{enumerate}

The forecast variable is most commonly called a random variable. In this case, according to the definition of flexibility outlined in Section \ref{Sect:ProblemStatement}, flexibility is considered as a random variable $X$. However, since we want to specify that we are considering one observation at a time, we will use the subscript $t$, resulting in the random variable $x_t$. This corresponds to the so-called time series data, meaning a set of observations $x_t$, each one being recorded at a specific time $t$. In this chapter and for the sake of simplicity, we will consider discrete time series, when observations are recorded under a discrete set, made at fixed time internvals. On the contrary, continuous time series are out of the scope of this research. 

When considering a probabilistic forecast, the forecast value $\hat{x}_t$ represents the average value of the forecast distribution.

Time series data have several characteristics that make their analysis different from other types of data: 
\begin{enumerate}
\item They can present a trend over time, understood as an increasing or decreasing offset tendency values in a given time series. 
\item The random variable may exhibit seasonality, understood as a fixed and known period. A seasonal pattern happens when a time series is influenced by seasonal factors such as the day of the week, the month of the year, among others. 
\item The data presents autocorrelation, meant as serial correlation between subsequent observations. 
\item The data might present cycles, when the data present rises and falls without a fixed frequency, most likely under economic conditions or cycles. 
\item The time series data presents an unexplanatory component, known as white noise. This can be understood as a random component that cannot be explained by any of the considered variables. This noise component is a stationary process with parameters mean and variance. 
\end{enumerate} 

An important part of the analysis of a time series is the choice of a suitable probability model for the data. The most common approach to deal with time series data is the implementation of autoregressive models (ARIMA/SARIMA). However, tuning the hyperparameters of this type of models can be a challenging task resulting in an algorithm that does not perform better than the benchmark \cite{nonlinearARIMA}, since ARIMA models assume linear functions of past data; and sometimes the time series data under study present non-linearities of one of the decomposed signals. Furthermore, time series decomposition to assume stationarity is mostly not feasible in the energy sector, making it more challenging to forecast time series under auto-regressive models. In the recent years, the most used approaches for forecasting time series data have been boosting algorithms and probabilistic forecast approaches \cite{Robinzonov2010, Barrow2016, Taieb2017, Guen2020}.  

\subsection{Benchmarks models}

The benchmark models are a helpful tool for setting the baseline of the performance of a forecast algorithm. This section covers the definition of two different benchmarks for the aggregated flexibility forecast task, the climatology model and the simple exponential smoothing (SES) model. Algorithm \ref{alg:climatology} presents the first benchmark developed for forecasting the flexibility available within an aggregator's portfolio. This model is based in the so-called na\"{i}ve, but evolved to consider the monthly seasonality in each month of the year. In this case, there is an average model created in each month of the year, represented by $m \in M$ and at each time period $t \in T$. As an example, the flexibility value for the 2nd of February at 10:00 consider all the previous flexibility values in February at 10:00, outlined as $y_{n|m,t}$,  providing the value $\hat{y}_{t+1|t}$. The set $N$ represents all the data points that belong to the same time period $t$ and month $m$, $n \in N | n \in  T \wedge M$. 

\begin{algorithm}
	\SetAlgoLined
	\KwIn{$Y$ observed data, $T, M$ time-related parameters}
	\KwResult{$\hat{Y}:\hat{y}_{t}$}
	\For{all $m \in M$}{	
		 \For{all $t \in T$} {
			$\hat{y}_{t+1|t} = \frac{1}{N} \ \mathlarger{\mathlarger{\sum}}_{n=1}^{N} \ \ \mathlarger{\mathlarger{y}}_{n|m,t}$ \\ 	 			}
	}
	\caption{Benchmark 1: Climatology Model}
	\label{alg:climatology}
\end{algorithm}

In the case of simple exponential smoothing, forecasts are calculated using weighted averages, meaning that there is a weight attached to each previous observation that decays exponentially as observations come from further in the past. As a result, the greatest weight is associated to the most recent observation, whereas the smallest weights correspond to the oldest observation. The smoothing parameter or time decay is represented by $\alpha$, being a constant value $\alpha \in [0,1]$. For the sake of clarification, those cases where $\alpha$ is close to 1, more importance and hence weight is given to more recent observations, On the contrary, when $\alpha$ is close to 0, more importance is provided to older observations. Algorithm \ref{alg:SES} outlines the setup for forecasting the flexibility signal based on historical data. 


\begin{algorithm}
	\SetAlgoLined
	\KwIn{$Y$ observed data, $T$ time granularity set, $\alpha$ decay factor}
	\KwResult{$\hat{Y}:\hat{y}_{t}$}
	\For{all $t \in T$}{
%		$\hat{y}_{t+1|t} = \mathlarger{\mathlarger{\sum}}_{n=1}^{T-1} $

     $\hat{y}_{t+1|t} = \mathlarger{\mathlarger{\sum}}_{n=1}^{T-1} \ \ \alpha \ (1-\alpha)^{n} \ y_{t-n} + (1-\alpha)^{t}\ell_{0}$\\
    }
	\caption{Simple Exponential Smoothing (SES)}
	\label{alg:SES}
\end{algorithm}


\subsection{Hierarchical model formulation}
The hierarchical model shown in Figure \ref{fig:top_down} has two different levels, being level~1 the characterization of whether there is flexibility available or not, represented by the random variable $X$, and level 2 the value of the available flexibility given the prior condition of availability, defined as a random variable $Y$. This yields
\begin{subequations}
\begin{align} 
\label{eq:hierarchicalpierre}
  & X \sim \mathcal{B}(p) \\
  & Y|X=1 \sim \mathcal{F}
\end{align}
\end{subequations}

In the first level of the hierarchy, the output value is either 0 or 1, following a Bernoulli distribution $X \sim \mathcal{B}(p)$ with associated probability $p$. The second level of the hierarchy aims to obtain the flexibility value assuming available flexibility or given that $X=1$. These data follow an unknown distribution named $\mathcal{F}$, which we will eventually approximate and track with Kernel Density Estimation (KDE). Consequently, the output of the model is obtained by combining the results of the two levels of the hierarchical model, as follows

\begin{equation} \label{eq:bernoulli}
  \mathbb{E}[Y] = \mathbb{E}[Y|X=1] \,  P[X=1]
\end{equation}
where the expected available flexibility at a specific time period is a result of multiplying the probability that flexibility is available (Level 1), times the expected value of flexibility given that the first condition is met (Level 2). The Root Mean Squared Error (RMSE) and the Mean Average Error (MAE) are chosen here as scores to evaluate the performance of the final outcome. The performance scores are calculated  at the end of the validation set, based on \cite{Hyndman2021}.

\subsection{Level 1: Bernoulli modeling for flexibility characterization}\label{sect:Level1}
Given the overview of the hierarchy, the first level is modeled according to a Bernoulli distribution $X \sim \mathcal{B}(p)$. This first level encodes the aggregated flexibility value into a binary signal, given a predefined threshold according to the characteristics of the flexible assets portfolio. Then, the random variable $X$  of the model can take a binary output either $k=0$ or $k=1$, with the associated probability $p$.

\begin{equation}
  P[X = k] =
  \begin{cases} 
    p     & \text{if $k = 1$}, \\
    1 - p & \text{if $k = 0$}.
  \end{cases}
\end{equation}


In order to determine the probability value $p$ at a specific time period in this first level of the hierarchy, we implemented a climatology model. This approach considers the flexibility binary states previous to that specific time and for a given month, resulting in the average value for $p$. The output of the model is then the average probability value,  $\overline{p}_{m,t}$,  for a time $t \in T$, for a day $d \in D$ and month $m \in M$, $p_{m,t} \in [0,1]$ and calculated as 

\begin{equation}
    \overline{p}_{m,t} = \frac{1}{D} \ \mathlarger{\sum_{d=1}^{D}} X_{d,m,t} \hspace{30pt} \forall m \in M, \ \forall t \in T
\end{equation}

This approach also provides valuable information for the input data required under the second level of the hierarchy (Figure \ref{fig:top_down}). In this case, the values lower than the threshold are removed from the dataset, obtaining the input data for Level 2. Accordingly, the resulting data and distribution are modeled according to an online and adaptive bandwidth KDE by means of Recursive Maximum Likelihood estimation.


\subsubsection{Model evaluation}

In order to evaluate the accuracy of a probabilistic prediction based on binary outcomes, we consider the Brier Score (BS). This evaluation method can be generally outlined as follows

\begin{equation}
    BS = \frac{1}{N} \sum_{t=1}^{N} (\overline{p}_{t} - o_{t})^2
\end{equation}


where $N$ is the total number of observations under the case study, $\bar{p}_{t}$ is the probability of the outcome to be 1, obtained at time $t$, and $o_{t}$ the binary outcome at time $t$.
\subsection{Level 2: Online KDE for flexibility value forecast}
% How we describe density 
Given that flexibility is available from the previous level of the hierarchy, the problem is now outlined by using a KDE, where a Gaussian Mixture Model (GMM) \cite{Silverman1986} of the observed data point is produced. Therefore, it is updated and adapted online based on new data samples fed into the model, similar to the approach outlined in \cite{Kristan2010,Pinson2012}. Formally, KDEs can be defined as
\begin{equation}
    f_{t}(y)= \frac{1}{n_{\lambda}} \ \sum_{i=1}^{t} \lambda^{t-i} \ \mathlarger{K}\left(\frac{y - y_{i}}{h_{t}}\right) 
\end{equation}

where $n_{\lambda} = \frac{1}{1-\lambda}$ is known as the equivalent window size and defines the number of observations used to calculate the updated flexibility value. The weight or forgetting factor can be defined as $\lambda$, being associated with that kernel. Accordingly, $h_{t}$ refers to the bandwidth of the kernel at time period $t$, $y$ is the vector of values where the function is evaluated, and $y_{i}$ is the measurement at time $t$, on which the kernel is going to be centered. Lastly, $\mathlarger{K}\left(\frac{y - y_{i}}{h_{t}}\right)$ is in this case a normalized Gaussian Kernel that can be formulated as

\begin{equation}
 \mathlarger{K}\left(\frac{y - y_{i}}{h_{t}}\right) = \frac{1}{h_{t} \sqrt{2 \pi}} \exp{\left(-\frac{1}{2}\left(\frac{y - y_{i}}{h_{t}}\right)^2\right) }
\end{equation}


Since the main objective of this approach is to adapt the resulting distribution as long as a new data point is fed into the model, an online learning approach is used. As a consequence, the kernel has to be updated at each time step in order to fit the new sample included in the model.

A uniform and normalized distribution is chosen as initial condition to start the recursive approach, and can be formulated as

\begin{equation}
    f_{t_{0}}(y) = \frac{1}{f_{max}}
\end{equation}

where $f_{max}$ is the maximum expected flexibility, considering all the available historical data at the beginning of the study. Hence, the kernel is updated at each time period by means of the following recursive formula 
\begin{equation} \label{eq:recursive_formula}
    f_{t}(y) = \lambda\ f_{t-1}(y) + (1-\lambda) \: \mathlarger{K}\left(\frac{y - y_{i}}{h_{t}}\right) \\
\end{equation}

which relies on the previous resulting distribution, together with the new data sample $y_{i}$ at time $t$ and associated KDE, $\mathlarger{K}\left(\frac{y - y_{i}}{h_{t}}\right)$. This methodology ensures that at each time step the normalized kernel properties are maintained since

\begin{equation}
  \int_{y} \mathlarger{K}\left(\frac{y - y_{i}}{h_{t}}\right) \, dy = 1 \\  
\end{equation}

\begin{equation}
  \int_{y} f_{t}(y) \, dy = 1  \quad \forall t \in T \\  
\end{equation}

%There are two parameters to estimate in this model, being the forgetting factor $\lambda$ and the kernel bandwidth for each time period ${h}_{t}$. The forgetting factor $\lambda$ is a real constant parameter in the range between 0 and 1, $\lambda\ \in\ (0,1]$. This value is estimated by means of cross-validation techniques. In contrast, the kernel bandwidth is estimated using Recursive Maximum Likelihood since it can be formulated to be time-adaptive, with good overall performance. 


\subsubsection{Adaptive Bandwidth estimation}
Since the main objective of this approach is to adapt the resulting distribution as long as a new data point is fed into the model, an online learning approach is used. As a consequence, the kernel has to be updated at each time step. To do so, the value of the bandwidth has to be adaptive in order to fit the new sample included in the model.

There are two parameters to estimate in this model, being the forgetting factor $\lambda$ and the kernel bandwidth for each time period $h_{t}$. The forgetting factor $\lambda$ is a real constant parameter in the range between 0 and 1, $\lambda\ \in\ [0,1)$. In the case of $h_{t}$, its value is defined as $h_{t} \in \mathbb{R}^+$, and calculated at each time step as a function of $y_{t}$. This yields 

% shows the formulation to estimate the value of $h_{t}$ $\forall t$. 

\begin{equation} \label{eq:ht}
    h_{t} = \widetilde{h} \, \sqrt{y_{t}}
\end{equation}

where $\widetilde{h}$ is an estimated and constant value $\widetilde{h} \in \mathbb \geqslant \, 0$. This function ensures that the resulting KDE does not provide a negative parameter for the flexibility estimation, according to the definition of flexibility defined in Section \ref{Sect:ProblemStatement}. In order to determine the value of $\widetilde{h}$ and $\lambda$ so as to generalize the model, grid search and cross-validation techniques have been implemented, using a cross-validation set of 3 months.

%A uniform and normalized distribution is chosen as initial condition to start the recursive formula. This distribution can be formulated as 
%
%\begin{equation}
%    f_{0}(y) = \frac{1}{f_{max}}
%\end{equation}
%
%where $f_{max}$ is the maximum expected flexibility, considering all the available historical data at the beginning of the study. Hence, the kernel is updated at each time period by means of the following recursive formula 
%\begin{equation} \label{eq:recursive_formula}
%    f_{t}(y) = \lambda\ f_{t-1}(y) + (1-\lambda) \: \mathlarger{K}\left(\frac{y - y_{i}}{h_{t}}\right) \\
%\end{equation}
%
%which relies on the previous resulting distribution, together with the new data sample and associated KDE. This methodology ensures that at each time step the normalized kernel properties are maintained since
%
%\begin{equation}
%  \int_{y} \mathlarger{K}\left(\frac{y - y_{i}}{h_{t}}\right) \, dy = 1 \\  
%\end{equation}
%
%\begin{equation}
%  \int_{y} f_{t}(y) \, dy = 1  \quad \forall t \\  
%\end{equation}

The setup used for the second level of the hierarchy used to calculate the resulting available flexibility value can be found in Algorithm \ref{alg:adaptive_bandwidth}. There, the previously defined formulation is structured together with the required input data, as well as the mathematical formulation in a generalized approach.


\begin{algorithm}
	\SetAlgoLined
	\KwIn{ $Y|X=1 \sim \mathcal{F}, \lambda, \tilde{h}$}
	\KwResult{$f_t (y) \; \forall t \in T$}
	at $t_{0}$ $\rightarrow$ \: $f_{t_{0}}(y) = \frac{1}{f_{max}} $\\
	\For{$ \forall\ t\ \in T $}{	
	$y_i$: read input data point at time $t$ \\
	$h_{t} = \widetilde{h} \, \sqrt{y_{i}}$ \\
	$f_{t}(y)= \frac{1}{n_{\lambda}} \ \sum_{i=1}^{t} \lambda^{t-i} \
\mathlarger{K}\left(\frac{y - y_{i}}{h_{t}}\right) $ \\
	$\mathlarger{K}\left(\frac{y - y_{i}}{h_{t}}\right) = \frac{1}{h_{t} \sqrt{2 \pi}} \exp{\left(-\frac{1}{2}\left(\frac{y - y_{i}}{h_{t}}\right)^2\right) }$ \\
	$f_{t}(y) = \lambda\ f_{t-1}(y) + (1-\lambda) \: \mathlarger{K}\left(\frac{y - y_{i}}{h_{t}}\right)$ 
	}
	\caption{Online Adaptive Bandwidth KDE}
	\label{alg:adaptive_bandwidth}
\end{algorithm}


\subsubsection{Recursive Maximum Likelihood for bandwidth estimation}

Let ${h}_{t}, \ t=1,...,T$ be the kernel bandwidth for a given time period $t$ of T time steps. This parameter is now estimated, $\hat{h}_{t}$, using a recursive approach, maximizing the likelihood, also known as Recursive Maximum Likelihood (RML). For convenience, the problem is formulated instead as a minimization problem, minimizing the log-likelihood, approach implemented as well in \cite{Pinson2009, Pinson_Madsen_2012}. Hence, $\hat{h}_{t}$ is going to get the value where the objective function is at its minimum for each time period $t$, at a given point $y_i$. This yields

\begin{equation}
    \hat{h}_{t} = \argmin_{h_{t}} \mathlarger{S}_t (h_{t}) 
\end{equation} 

In this case, the objective function $S_t(h_{t})$ to be minimized at each time period is a function of $h_{t}$, and it can be formulated as follows 

\begin{equation}
    \mathlarger{S}_t (h_{t}) = -\frac{1}{n_{\lambda}}\ \sum_{i=1}^{t} \ \lambda^{t-i} \ \ln f_t(y_i)
\end{equation}

\begin{equation}
    \mathlarger{S}_t (h_{t}) = \lambda \ \mathlarger{S}_{t-1}(h_{t}) - (1-\lambda) \  \ln f_t(y_i)
\end{equation}

We define a recursive estimation procedure for calculating $\hat{h}_{t}$. In this case, we implement a Newton-Raphson step to express the estimation of $\hat{h}_{t}$, defined as $\hat{h}_t \in \mathbb{R}^+$, as a function of the previous estimation. 
The bandwidth of the estimated kernel must be positive, since the flexibility value is defined positive according to Section \ref{Sect:ProblemStatement}. However, the mathematical formulation of the model can lead to negative bandwidths. Hence, a logarithmic transformation is included in the model to ensure that the bandwidth is always positive. The new parameter defined is $\tilde{h}_t \in \mathbb{R}$. This yields

\begin{equation}
    \tilde{h}_t = \tilde{h}_{t-1} - \frac{ \nabla_h \ \mathlarger{S}_t (\hat{h}_{t-1}) }{\nabla^2_h \ \mathlarger{S}_t (\hat{h}_{t-1})}
\end{equation}

\begin{equation}
    \hat{h}_t = e^{\tilde{h}_t}
\end{equation}

To compute the estimated value for $\hat{h}_{t}$, we calculate the derivative terms $ \nabla_h \ \mathlarger{S}_t (\hat{h}_{t})$ and $\nabla^2_h \ \mathlarger{S}_t (\hat{h}_{t})$ which can be expressed as
\begin{equation}
     \nabla_h \ \mathlarger{S}_t (\hat{h}_{t}) =  \lambda\underbrace{\nabla_h \ \mathlarger{S}_{t-1} (\hat{h}_{t})}_{\text{= 0, optimal state}} - \frac{1}{n_{\lambda}} \frac{\nabla f_t (y)}{f_t (y)}
\end{equation}

Where the first term is equal to 0, since we assume that we were under the optimal state at $t-1$. This gives the formal solution

\begin{equation}
     \nabla_h \ \mathlarger{S}_t (\hat{h}_{t}) =  (\lambda - 1) \ \mathlarger{U}_t
\end{equation}

Where $U_t$ is defined as the information vector

\begin{equation}
     U_t =  \frac{\nabla _h \, f_t (y)}{f_t (y)}
\end{equation}

In which the numerator can be outlined as follows
\begin{equation}
    \nabla_{h}\, f_t (y) = \lambda \ \nabla_{h}\,f_{t-1}(y) + (1-\lambda) \ \left(\frac{(y - y_{i})^2}{\hat{h}_{t}^2} - 1\right) \cdot \mathlarger{K}\left(\frac{y - y_{i}}{\hat{h}_{t}}\right)
\end{equation}


Accordingly, the term $\nabla^2_h \ \mathlarger{S}_t (\hat{h}_{t})$ can be written as 

\begin{equation}
    \nabla^2_h \ \mathlarger{S}_t (\hat{h}_{t}) = \lambda \ \nabla^2_h \ \mathlarger{S}_{t-1} (\hat{h}_{t}) - \frac{1}{n_{\lambda}} \frac{d}{dh_{t}} \left(\frac{\nabla_h f_t (y)}{f_t (y)}\right)
\end{equation}
\begin{equation}
      \nabla^2_h \ \mathlarger{S}_t (\hat{h}_{t}) = \lambda \nabla^2_h \ \mathlarger{S}_{t-1} (\hat{h}_t) - \frac{1}{n_{\lambda}} \frac{\overbrace{\nabla^2_h f_t (y) \cdot f_t (y)}^{\text{$\approx$ 0}} - \nabla_h f_t (y) \cdot \nabla_h f_t (y) }{f_t (y) ^2}
\end{equation}

The first and over-braced term can be neglected according to equations (28) and (29) in \cite{Pinson_Madsen_2012}. There, we assume that $f_t$ is almost linear in the close vicinity of $\hat{h}_t$ for a given $t \in T$. Thus, it can be outlined as 

\begin{equation}
    \frac{\nabla^2_h f_t (y) \cdot f_t (y)}{f_t (y)^2} \ll \frac{- \nabla_h f_t (y) \cdot \nabla_h f_t (y) }{f_t (y) ^2}
\end{equation}

Hence, it can be translated into 

\begin{equation}
    \frac{\nabla^2_h f_t (y)}{f_t (y)} \ll \frac{- \nabla_h f_t (y)^2 }{f_t (y) ^2}
\end{equation}

As a result,

\begin{equation}
\frac{d}{dh} \left(\frac{\nabla_h f_t (y)}{f_t (y)}\right) \simeq  \frac{- \nabla_h f_t (y) \cdot \nabla f_t (y) }{f_t (y) ^2}    
\end{equation}
\begin{equation}
\frac{d}{dh} \left(\frac{\nabla_h f_t (y)}{f_t (y)}\right) \simeq  - \left(\frac{\nabla f_t (y)}{f_t (y)} \right)^2  
\end{equation}

Hence, this allows a formal solution to be found as 

\begin{equation}
    \nabla^2_h \ \mathlarger{S}_t (\hat{h}_{t}) = \lambda \ \nabla^2_h \ \mathlarger{S}_{t-1} (\hat{h}_t) + (1-\lambda) \left(\frac{\nabla_h f_t (y)}{f_t (y)} \right)^2  
\end{equation}
\begin{equation}
     \nabla^2_h \ \mathlarger{S}_t (\hat{h}_{t}) = \lambda \ \nabla^2_h \ \mathlarger{S}_{t-1} (\hat{h}_t) + (1-\lambda) \ \mathlarger{U}_t^2
\end{equation}

This formulation can be implemented in different time-scales, being for example a Single Model for the hourly flexibility estimation, created at the initial time period and recursively updated at each time period $t \in T$. Another implementation could be a multiple hourly model, defined as Hourly Model, where the flexibility estimation in each hour is characterized by a density function, and updated at that specific hourly time period $t \in T$ for each day. Both approaches are implemented and analyzed based on the case study data in Section \ref{sect:Results}.

The setup used in this section can be found in Algorithm \ref{alg:RML_algorithm}. There, the previously defined formulation is structured together with the required input data, as well as the mathematical formulation in a generalized approach. It is worth to consider a number of particularities in the following algorithm to ensure the good performance of the code for all data points. A lower bound has been included in the model by means of a tolerance value.  This is done to avoid discontinuities in the calculation of the information vector (line 4 in Algorithm~\ref{alg:RML_algorithm}) when the read data point $y_{i}$ is closer to the tails of the approximated distribution. However, for the sake of clarity, this is not include in the Algorithm formulation, but can be found in the code source. Both terms $\nabla_h \, \mathlarger{S}_t (\hat{h}_{t})$ and $\nabla^2_h \ \mathlarger{S}_t (\hat{h}_{t})$ are considered as a function of $h_{t}$. These functions are evaluated at each time period $t \in T$ at a vector of specific data points $y$ defined at $t_0$. Nevertheless, in order to update the estimated value of the bandwidth $\hat{h}_t$, $\nabla_h \, \mathlarger{S}_t (\hat{h}_{t})$ and $\nabla^2_h \ \mathlarger{S}_t (\hat{h}_{t})$ have to be evaluated and hence interpolated in a new data point, being in this case $y_i$ (lines 6 and 7). Finally, the if-statement in line 8 considers a warm-start initialization, represented as $t_{ws}$.

%\begin{algorithm}[]
%\caption{Online KDE using Recursive Maximum Likelihood}
%\begin{spacing}{1.7}
%\hspace*{\algorithmicindent} \textbf{Input data} $Y|X=1 \sim \mathcal{F}$ 
%\begin{algorithmic}[1] \label{alg:RML_algorithm}
%%\STATE     $f_{t}(y)= \frac{1}{n_{\lambda}} \ \sum_{i=1}^{t} \lambda^{t-i} \
%%\mathlarger{K}\left(\frac{y - y_{i}}{h_{t}}\right) $
%%\STATE $\mathlarger{K}\left(\frac{y - y_{i}}{h_{t}}\right) = \frac{1}{h_{t} \sqrt{2 \pi}} \exp{\left(-\frac{1}{2}\left(\frac{y - y_{i}}{h_{t}}\right)^2\right) }$
%\STATE at $t_{0}$ $\rightarrow$ \: $f_{t_{0}}(y) = \frac{1}{f_{max}}, \: df_{t_{0}}(y) = 0, \: \nabla^2_h \mathlarger{S}_{t_{0}}(y) = \frac{1}{f_{max}},\: \tilde{h}_{t_{0}} = -1$ \\ %initialization of fy, initialization of dfy, initialization of hessian S, initialization of hh (h_tilde) 
%%\STATE \hspace{40pt} $\hat{h}_{t_{0}} = e^{\tilde{h}_{t_{0}}}$ \\ %initialization of hy (h_hat)
%\FOR { $ \forall\ t\ \in T $} 
%     \STATE $y_i$: read input data point at time $t$ 
%     \STATE $U_t = \frac{\nabla_h \, f_t (y)}{f_t (y)}$\\ %update information vector
%     \STATE $\nabla_h \, \mathlarger{S}_t (\hat{h}_{t-1}) =  (\lambda - 1) \ \mathlarger{U}_t$ \\ %gradient update\\
%     \STATE \text{$\nabla_h \, \mathlarger{S}_t(\hat{h}_{t-1},\ y_i)$: \ retrieve the value through linear interpolation of $\nabla_h \mathlarger{S}_t(\hat{h}_{t})$ } \\ \hspace{75pt} and $y$ in the read data point $y_i$\\
%     \STATE \text{$\nabla^2_h \, \mathlarger{S}_t(\hat{h}_{t-1},\ y_i)$: \ retrieve the value by linear interpolation of $\nabla^2_h \mathlarger{S}_t(\hat{h}_{t})$} \\ \hspace{75pt} and $y$ in the read data point $y_i$\\
%     %\STATE \text{$\nabla_h \mathlarger{S}_t(\tilde{h}_{t-1},\ y_i)$, $\nabla^2_h \mathlarger{S}_t(\tilde{h}_{t-1},\ y_i)$: \ retrieve the values through linear interpolation} \\  \hspace{145pt} of $\nabla_h \mathlarger{S}_t(\tilde{h}_{t})$ and $\nabla^2_h \mathlarger{S}_t(\tilde{h}_{t})$ and $y$ in a \\ \hspace{145pt} new data point $y_i$\\
%          \IF {$t\geq t_{ws}$}  % warm start implementation
%                \STATE $\tilde{h}_t = \tilde{h}_{t-1} - \frac{ \nabla_h \ \mathlarger{S}_t (\hat{h}_{t-1},\ y_i) }{\nabla^2_h \ \mathlarger{S}_t (\hat{h}_{t-1},\ y_i)}$
%        \ENDIF 
%     \STATE $\hat{h}_t = e^{(\tilde{h}_{t})}$ %compute hy based on hh (np.exp)\\
%     \STATE  $f_{t}(y) = \lambda\ f_{t-1}(y) + (1-\lambda) \: \mathlarger{K}\left(\frac{y - y_{i}}{\hat{h}_{t}}\right)$ \\ %Recursive formula for fy \\
%     \STATE $\nabla_h \, f_t (y) = \lambda \ \nabla_h \,f_{t-1}(y) + (1-\lambda) \ \left(\frac{(y - y_{i})^2}{\hat{h}_{t}^2} - 1\right) \cdot \mathlarger{K}\left(\frac{y - y_{i}}{\hat{h}_{t}}\right) $ \\ %Recursive formula for dfy\\
%     \STATE $\nabla^2_h \ \mathlarger{S}_t (\hat{h}_{t}) = \lambda \ \nabla^2_h \ \mathlarger{S}_{t-1} (\hat{h}_t) + (1-\lambda) \ \mathlarger{U}_t^2$\\ %Recursive formula for  hessian\\
%\ENDFOR
%\end{algorithmic} 
%\end{spacing}
%\end{algorithm}


\begin{algorithm}
	\SetAlgoLined
	\KwIn{ $Y|X=1 \sim \mathcal{F}$ }
	\KwResult{$f_t (y) \; \forall t \in T$}
	at $t_{0}$ $\rightarrow$ \: $f_{t_{0}}(y) = \frac{1}{f_{max}}, \: df_{t_{0}}(y) = 0, \: \nabla^2_h \mathlarger{S}_{t_{0}}(y) = \frac{1}{f_{max}},\: \tilde{h}_{t_{0}} = -1$ \\
	
	\For{$ \forall\ t\ \in T $}{	
	$y_i$: read input data point at time $t$ \\
	$U_t = \frac{\nabla_h \, f_t (y)}{f_t (y)}$\\ %update information vector
	$\nabla_h \, \mathlarger{S}_t (\hat{h}_{t-1}) =  (\lambda - 1) \ \mathlarger{U}_t$ \\ %gradient update\\
	\text{$\nabla_h \, \mathlarger{S}_t(\hat{h}_{t-1},\ y_i)$: \ retrieve the value through linear interpolation of} \\ \hspace{75pt}  $\nabla_h \mathlarger{S}_t(\hat{h}_{t})$  and $y$ in the read data point $y_i$\\
    \text{$\nabla^2_h \, \mathlarger{S}_t(\hat{h}_{t-1},\ y_i)$: \ retrieve the value by linear interpolation of} \\ \hspace{75pt} $\nabla^2_h \mathlarger{S}_t(\hat{h}_{t})$ and $y$ in the read data point $y_i$\\		 
	          \eIf{$t\geq t_{ws}$}{  % warm start implementation
				$\tilde{h}_t = \tilde{h}_{t-1} - \frac{ \nabla_h \ \mathlarger{S}_t (\hat{h}_{t-1},\ y_i) }{\nabla^2_h \ \mathlarger{S}_t (\hat{h}_{t-1},\ y_i)}$
				}{
	$\hat{h}_t = e^{(\tilde{h}_{t})}$\\ %compute hy based on hh (np.exp) 
	$f_{t}(y) = \lambda\ f_{t-1}(y) + (1-\lambda) \: \mathlarger{K}\left(\frac{y - y_{i}}{\hat{h}_{t}}\right)$ \\ %Recursive formula for fy \\
	$\nabla_h \, f_t (y) = \lambda \ \nabla_h \,f_{t-1}(y) + (1-\lambda) \ \left(\frac{(y - y_{i})^2}{\hat{h}_{t}^2} - 1\right) \cdot \mathlarger{K}\left(\frac{y - y_{i}}{\hat{h}_{t}}\right) $ \\ %Recursive formula for dfy\\
	$\nabla^2_h \ \mathlarger{S}_t (\hat{h}_{t}) = \lambda \ \nabla^2_h \ \mathlarger{S}_{t-1} (\hat{h}_t) + (1-\lambda) \ \mathlarger{U}_t^2$\\ %Recursive formula for  hessian\\ 
	}   
	}
	\caption{Online KDE using Recursive Maximum Likelihood}
	\label{alg:RML_algorithm}
\end{algorithm}


\subsubsection{Model evaluation}
This approach has the forgetting factor or time decay $\lambda$ as a hyper-parameter to be tuned. The choice of an optimal value for $\lambda$ is implemented by using a cross-validation technique. In this case, the last three months of data are used as a validation set to validate the optimal value of the forgetting factor, out of the 6000 available hourly data points of flexible consumption under the second level of the hierarchy. For the evaluation of the probabilistic forecast, in this case a density distribution, we follow the approach of a log-likelihood score (LS). For a given predictive density distribution $f_{t}(y)$ and corresponding measured available flexibility value, written as $y_{i+1}$, the score can be formulated as

\begin{equation}
    LS_{t} = - \ln \left(f_{t}(y_{i+1})\right)
\end{equation}

Accordingly, the LS value can be averaged over the evaluation set, given by 

\begin{equation}
    LS = - \frac{1}{N_{CV}} \ \sum_{t=1}^{N_{CV}} \ln \left(f_{t}(y_{i+1})\right)
\end{equation}

where $N_{CV}$ represents the number of data points considered under the validation set.
The optimal value of the forgetting factor $\lambda$ is chosen as that which minimizes the log-likelihood score  over the validation set. 

\section{Case study} \label{Sect:CaseStudy}
The data used in this paper have been obtained from the Dataport Pecan Street Inc. dataset, collected from real households in Austin, Texas, the USA \cite{PecanStreetInc}. This dataset contains appliance-level energy consumption data from 25 households during one year of metering and sub-metering in 2018, providing the dataset in different granularity. Table \ref{table:dataport_austin} shows an overview of the dataset used for the model.


\begin{table}[htbp]
\centering
\refstepcounter{table}
\label{table:dataport_austin}
\resizebox{0.7\textwidth}{!}{%
\begin{tabular}{ll} 
\toprule
\textbf{Element}                           & \textbf{Description}    \\ 
\hline
Number of users          & 25                      \\
Location                & Austin, Texas (USA)     \\
Types of flexible assets & PV, ESS, EV, SH, EWB    \\
Data granularity         & 1 min - 5 min - 15 min  \\
\# users with PV panels  & 9 (36\%)                \\
\# users with ESS        & 0 (0\%)                 \\
\# users with SH         & 0 (0\%)                 \\
\# users with EWB        & 5 (20\%)                \\
\# users with EV         & 4 (16\%)                \\
\bottomrule
\end{tabular}
}
\end{table}

The final dataset has been defined considering the following particularities: (i) Specific types of flexible loads and renewable generation features have been chosen, constituting the dataset: Photovoltaic panels $PV$, Energy Storage Systems $ESS$, Electric Water Boilers $EWB$, Space Heaters $SH$, and Electric Vehicles $EV,$ being the energy of each of them measured in kWh and aggregated according to the bottom-up approach shown in Figure \ref{fig:bottom_up}; (ii) The total, the flexible and the inflexible load signals are then formulated as an aggregation of the flexible assets for each user $i$, at a given time $t$, and also considering the net load $Net$ available in the raw dataset. This can be formulated as follows 
\begin{subequations}
\begin{align} 
  & Total_{t} = \sum_{i=1}^{N} (Net_{i,t} - ESS_{i,t} - PV_{i,t}) \\
  & Flex_{t} = \sum_{i=1}^{N} (EWB_{i,t} +  SH_{i,t} + EV_{i,t}) \\
  & Inflex_{t} = Total_{t} - Flex_{t}
\end{align}
\end{subequations}

Figure \ref{fig:agg_load} displays a sample of the aggregated total, flexible and inflexible load of the dataset used in the case study, over an arbitrarily chosen episode of one week. 

\begin{figure}[htbp]
\centerline{\includegraphics[width=0.8\columnwidth]{ChapterAggFlexForecast/Figures/total_flex_inflex_FINAL_cropped.png}}
\caption{Aggregated total, flexible and inflexible loads}
\label{fig:agg_load}
\end{figure}

Open-access datasets with real-world data containing metering and submetering measurements are not generally accessible. As a result, the available dataset has some limitations such as that there are not any users with SH. This can be understood as a limitation of the model but also as a representation of a real dataset, where not all end-users have flexible assets, and where SH are not commonly used.  Likewise, having only one year of data and a portfolio of 25 users can lead to shortcomings in the flexibility modeling. PV generation and ESS are not defined as flexible loads but as flexible prosumption, and are not considered into the flexibility forecast at this stage of the research, because these assets are defined as generation or generation/consumption signals. Hence, it can lead to a misunderstanding of the signal itself based on the definition of flexibility considered in Section \ref{Sect:ProblemStatement}. 

%\subsection{Data availability}
%The case study related data cannot be shared due to license restrictions. Nevertheless, the code and a different data sample regarding the Online Recursive Maximum Likelihood Kernel Density Estimation approach implemented in this chapter can be found at \url{http://dx.doi.org/10.17632/t92mmtm4gs.1}, hosted at Mendeley Data \cite{Munne-collado_dataset} for the sake of reproducibility of the results.

\section{Analysis of results} \label{sect:Results} 
This section presents the results obtained after applying the hierarchical model described in Section \ref{Sect:FlexModeling} to the case study presented in Section \ref{Sect:CaseStudy}. Section \ref{Sect:ResultsLevel1} corresponds to the output of the first level of the hierarchy: the flexibility availability forecast. Section \ref{Sect:ResultsLevel2AdaptiveBandwidth} shows the output of the second level of the hierarchy, being it the flexibility value estimation using the adaptive bandwidth approach. In Section \ref{Sect:ResultsLevel2SingleModel} two different strategies are implemented: a single model, created at the first instant and updated for each time period of the time horizon considered; and an hourly model, created for each time period of the day and updated daily. In section \ref{Sect:ResultsFinalOutcome}, the first and second levels of the hierarchy are combined to obtain the expected flexibility value. All data preparation, processing, and model simulations were carried out using a desktop unit with an Intel Core i7-10510U quad-core CPU @  1.8-4.9~GHz with 16 GB RAM.

\subsection{Flexibility availability forecast: Hierarchy level~1} \label{Sect:ResultsLevel1}
In the first level of the hierarchy, the defined threshold for encoding the dataset has been $0.20$ kWh, resulting in a dataset with one year of data encoded within binary outputs 0-1. Based on that, the climatology model has been computed and evaluated. 
Figure \ref{fig:LEVEL1-P_YEAR} shows the probability value of having available flexibility, represented by the binary output $k=1$, for a given time period $t$ under the first level of the hierarchical model. For the sake of simplicity, we have stratified the probability value based on seasonality. Results show that, regardless of the season, afternoon and evening time periods from 14:00 until 21:00 is where the greatest probability of available flexibility is provided, following similar patterns. Despite this, it can also be seen that there are differences in specific time periods and seasons, being fall the season with the lowest flexibility availability until 6:00 in the morning. At the same time, fall is the season where the highest flexibility availability is seen in the afternoon, and at the same time achieving greater values earlier in the afternoon, compared to the other seasons.   

\begin{figure}[htbp]
\centerline{\includegraphics[width=0.8\columnwidth]{ChapterAggFlexForecast/Figures/probability_season_final_cropped.png}}
\caption{Available flexibility. Probability value based on season and time period}
\label{fig:LEVEL1-P_YEAR}
\end{figure}

The Brier Score (BS) is formulated as $\in \mathbb{R}^+ [0,1]$, assuming 0 as the perfect forecast and 1 the worst forecast, and it is considered as the evaluation method for the model developed under the first level of the hierarchy. Table \ref{tab:level1-scores} provides an overview of the results, where the BS obtained is $0.196$. That means that the probabilistic forecast calculated by this methodology can be considered as accurate enough. Furthermore, the computational time for which this algorithm computes the probabilities for one year of data is less than one minute, making this algorithm fast enough for operational purposes where flexibility must be estimated. 

%\begin{table}[]
%\centering
%\caption{Results of model evaluation procedure for the climatology model in the first level of the hierarchy, using the Brier Score (BS) as a performance score.}
%\label{tab:level1-scores}
%\resizebox{0.7\textwidth}{!}{%
%\begin{tabular}{c|c|c|}
%\cline{2-3}
% &
%  \textbf{\begin{tabular}[c]{@{}c@{}}BS {[}-{]}\end{tabular}} &
%  \textbf{\begin{tabular}[c]{@{}c@{}}Processing time {[}s{]}\end{tabular}} \\ \hline
%\multicolumn{1}{|l|}{\textbf{Level 1 - Single Model}} & 0.196
%   & 53.31   \\ \hline
%\end{tabular}%
%}
%\end{table}

\begin{table}[htbp]
\centering
\caption{Results of model evaluation procedure for the climatology model in the first level of the hierarchy, using the Brier Score (BS) as a performance score.}
\vspace*{3mm}
\label{tab:level1-scores}
\resizebox{0.7\textwidth}{!}{%
\begin{tabular}{ccc} 
\toprule
                                                    & \textbf{BS [-]} & \textbf{Processing time [s]}  \\ 
\hline
\multicolumn{1}{l}{\textbf{Level 1 - Single Model}} & 0.196           & 53.31                         \\
\bottomrule
\end{tabular}
}
\end{table}

\subsection{Flexibility value estimation: Hierarchy level 2 - Adaptive bandwidth estimation}\label{Sect:ResultsLevel2AdaptiveBandwidth}

As a starting point for the development of a probabilistic forecast model for the second level of the hierarchy, the formulation of the kernel density estimation model based on the adaptive bandwitdh estimation is applied to the study case.
Fig. \ref{fig:KDE_EVOLUTION1} shows a representation of the online KDE algorithm for a specific time period. This visualisation aims to highlight the performance of the algorithm as new data are fed into the model, calculating a KDE for each new data point; and updating the actual distribution based on the obtained parameters $\widetilde{h}$ and $\lambda$. 

\begin{figure}[!ht]
\centerline{\includegraphics[width=0.8\columnwidth]{ChapterAggFlexForecast/Figures/KDE_evolution_final2_cropped.png}}
\caption{Online KDE algorithm on February 13, 2018, 16:00}
\label{fig:KDE_EVOLUTION1}
\end{figure}

With the aim of evaluating the performance of the online and adaptive bandwidth KDE, the Root Mean Square Error (RMSE) and the log-likelihood score have been chosen as a metric to measure the accuracy of the model. Fig. \ref{fig:grid_search_hyperparameters_RMSE} shows the RMSE value of the online KDE algorithm, depending on the values taken by $\widetilde{h}$ and $\lambda$. These results have been obtained under grid search and cross-validation algorithms, with the objective to define the best combination of $\widetilde{h}$ and $\lambda$ to achieve the minimum RMSE value. The same methodology is implemented for calculating the best parameters of the model using the LS as the evaluation method, shown in Figure \ref{fig:grid_search_hyperparameters_LS}. We first considered a broader range for $\lambda$ and $\widetilde{h}$ under the grid search and cross-validation techniques. Therefore a second grid search was performed, narrowed around the minimum value for both the RMSE and LS score. According to the results observed in Fig. \ref{fig:grid_search_hyperparameters_RMSE}, the minimum RMSE value of the model under the train set is RMSE $= 0.0458$, with $\widetilde{h}=0.7555$ and $\lambda=0.6455$. In the case of evaluating the model with the LS evaluation method, the optimal values found are $\widetilde{h}=0.4$ and $\lambda=0.978$, resulting in a LS $=2.15$ under the train set.

%Since the real distribution is unknown, the RMSE is calculated against the histogram, being that the most realistic available approximation to the real distribution. The histogram has been calculated using the entire dataset and using the Freedman-Diaconis rule to select the width of the bins. In the case of the log-likelihood score the score is calculated at each time period based on the true value of the read element. 

%
%\begin{figure}[!ht]
%\centerline{\includegraphics[width=0.6\columnwidth]{ChapterAggFlexForecast/Figures/surface3D_2ndGRIDSEARCH_cropped.png}}
%\caption{Representation of RMSE as a function of $\widetilde{h}$ and $\lambda$ grid search values. }
%\label{fig:grid_search_hyperparameters_LS}
%\end{figure}


\begin{figure}[htbp]
\centering     %%% not \center
\subfigure[Representation of RMSE as a function of $\widetilde{h}$ and $\lambda$ grid search values.]{\label{fig:grid_search_hyperparameters_RMSE}\includegraphics[width=60mm]{ChapterAggFlexForecast/Figures/surface3D_2ndGRIDSEARCH_cropped.png}}
\subfigure[Representation of LS as a function of $\widetilde{h}$ and $\lambda$ grid search values.]{\label{fig:grid_search_hyperparameters_LS}\includegraphics[width=60mm]{ChapterAggFlexForecast/Figures/3D_surface_loglikelihood_1.png}}
\caption{Grid-search results comparison}
\label{fig:onlinekde_rmse}
\end{figure}


The final distribution of the flexibility value can be calculated once the resulting best parameters for $\widetilde{h}$ and $\lambda$ have been obtained at the end of the test case set, as displayed in Figures \ref{fig:finaldensity_RMSE} and \ref{fig:finaldensity_ls}. This figure shows the final probability density function, obtained once the entire dataset has been included in the model under the cross-validation step, and can be compared to the histogram of the dataset under the second level of the hierarchy at that time period. Beware that the figure is displayed as a representation of the results; however, it is not directly comparable. The explanation is that the histogram is a static representation, whereas the resulting density functions are time adaptive based on the values of $\lambda$ and $\widetilde{h}$. It is interesting to highlight the differences between the resulting distribution for the combination of $\lambda$ and $\widetilde{h}$ with the most accurate performance.

\begin{figure}[ht!]
\centering     %%% not \center
\subfigure[Online KDE algorithm on February 13, 2018, 16:00 using the grid-search optimal parameters under RMSE evaluation.]{\label{fig:finaldensity_RMSE}\includegraphics[width=60mm]{ChapterAggFlexForecast/Figures/KDE_GS_final_cropped.png}}
\subfigure[Online KDE algorithm on February 13, 2018, 16:00 using the grid-search optimal parameters under LS evaluation.]{\label{fig:finaldensity_ls}\includegraphics[width=63mm]{ChapterAggFlexForecast/Figures/final_density_loglikelihood_onlineKDE_basic_2.png}}
\caption{Final density functions comparison}
\label{fig:distribution_RMSE_LS}
\end{figure}

The approximated distribution, obtained with the online KDE, follows the histogram of the dataset. However, our results were unsatisfactory in terms of the approximated distribution when the flexibility value is around 3.75 kWh, being the resulting distribution not capable of following the peak value on both cases under the RMSE and LS score. These differences can be accounted for as a limitation of the model since the best $\lambda$ value obtained under the grid search and cross-validation methods using RMSE as the evaluation method can be considered excessively low, leading to a significant down-weighting of the previous values when the distribution is updated at each time period $t$. In the case of using LS for finding the best parameters of the model, there is a better performance and the resulting $\lambda$ prevents the model from forgetting the previous observations too fast. However, by looking at the resulting distribution, it can be concluded that the adaptive bandwidth formulation should be improved in order to be shaped according to non-parametric distributions with more that one mode. This is why this method is not shown as a combination of levels. Consequently, this model is improved by means of the recursive maximum likelihood approach for determining the hyperparameters. The results of the recursive maximum likelihood approach are shown in the following section.  

%\begin{figure}[!ht]
%\centerline{\includegraphics[width=0.6\columnwidth]{ChapterAggFlexForecast/Figures/KDE_GS_final_cropped.png}}
%\caption{Resulting distribution for the best two combinations of  $\lambda$ and $\widetilde{h}$ at the end of the test case set, June 19, 2018, 09:00}
%\label{fig:KDE_GS}
%\end{figure}



%\begin{figure}[htbp]
%\centering     %%% not \center
%\subfigure[Representation of RMSE as a function of $\widetilde{h}$ and $\lambda$ grid search values.]{\label{fig:grid_search_hyperparameters_RMSE}\includegraphics[width=60mm]{ChapterAggFlexForecast/Figures/surface3D_2ndGRIDSEARCH_cropped.png}}
%\subfigure[Resulting distribution for the best two combinations of  $\lambda$ and $\widetilde{h}$ at the end of the test case set, June 19, 2018, 09:00]{\label{fig:KDE_EVOLUTION1}\includegraphics[width=60mm]{ChapterAggFlexForecast/Figures/KDE_GS_final_cropped.png}}
%\caption{Results overview with RMSE as evaluation method}
%\label{fig:onlinekde_rmse}
%\end{figure}


%\begin{figure}[htbp]
%\centering     %%% not \center
%\subfigure[Representation of RMSE as a function of $\widetilde{h}$ and $\lambda$ grid search values.]{\label{fig:grid_search_hyperparameters_RMSE}\includegraphics[width=60mm]{ChapterAggFlexForecast/Figures/surface3D_GRIDSEARCH_cropped.png}}
%\subfigure[Representation of LS as a function of $\widetilde{h}$ and $\lambda$ grid search values.]{\label{fig:grid_search_hyperparameters_LS}\includegraphics[width=60mm]{ChapterAggFlexForecast/Figures/3D_surface_loglikelihood_1.png}}
%\caption{Grid-search results comparison}
%\label{fig:onlinekde_rmse}
%\end{figure}


%
%\begin{figure}[!ht]
%\centerline{\includegraphics[width=0.6\columnwidth]{ChapterAggFlexForecast/Figures/3D_surface_loglikelihood_1.png}}
%\caption{Representation of the LS score as a function of $\widetilde{h}$ and $\lambda$ grid search values.}
%\label{fig:surface_LS}
%\end{figure}


%\begin{figure}[!ht]
%\centerline{\includegraphics[width=0.6\columnwidth]{ChapterAggFlexForecast/Figures/final_density_loglikelihood_onlineKDE_basic.png}}
%\caption{Online KDE algorithm on February 13, 2018, 16:00 using the grid-search optimal parameters under LS evaluation.}
%\label{fig:finaldensity_ls}
%\end{figure}



%\begin{figure}[htbp]
%\centering     %%% not \center
%\subfigure[Representation of LS as a function of $\widetilde{h}$ and $\lambda$ grid search values.]{\label{fig:grid_search_hyperparameters_LS}\includegraphics[width=60mm]{ChapterAggFlexForecast/Figures/3D_surface_loglikelihood_1.png}}
%\subfigure[Representation of LS as a function of $\widetilde{h}$ and $\lambda$ grid search values.]{\label{fig:finaldensity_ls}\includegraphics[width=60mm]{ChapterAggFlexForecast/Figures/final_density_loglikelihood_onlineKDE_basic.png}}
%\caption{Online KDE algorithm on February 13, 2018, 16:00 using the grid-search optimal parameters under LS evaluation.}
%\label{fig:onlinekde_LS}
%\end{figure}


\subsection{Flexibility value estimation: Hierarchy level 2 - Online RML-KDE performance}\label{Sect:ResultRMLalgorithm}

Prior to the visualization and explanation of the results obtained by implementing the online RML-KDE formulation to the case study, this section aims to validate the presented mathematical formulation presented in Algorithm \ref{alg:RML_algorithm} under a normal distribution with parameters $\mu = 0$ and $\sigma = 1$, and $1 \times 10^{6}$ random samples. As shown in Figure \ref{fig:rml_evolution} this online learning algorithm updates the resulting distribution.  More specifically, Figure \ref{fig:it50} outlines the performance of the algorithm in the 50$^{th}$ iteration and Figure \ref{fig:it19500} under iteration 19500. It can be seen that the first iteration starts with a uniform distribution, and that by iteration 50 the resulting distribution starts to be shaped. Since the resulting distribution needs a significant amount of iterations to be shaped and provide accurate forecast, the resulting algorithm presents a warm-start to initialize the model before computing the evaluation score.

\begin{figure}[ht!]
\centering     %%% not \center
\subfigure[Iteration 50]{\label{fig:it50}\includegraphics[width=62mm]{ChapterAggFlexForecast/Figures/RML_IT55.png}}
\subfigure[Iteration 19500]{\label{fig:it19500}\includegraphics[width=62mm]{ChapterAggFlexForecast/Figures/RML_IT19000.png}}
\caption{Online RML-KDE algorithm evolution}
\label{fig:rml_evolution}
\end{figure}

Once the model is validated, it is implemented under the case study. In this case, before tuning the hyperparameters of the model, it is important to highlight the importance of the time-decay factor $\lambda$ in the resulting distribution. $\lambda$ can act as a smoothing factor of the resulting distribution; the lower $\lambda$ is, the noisier the resulting function is. Accordingly, the larger the value of $\lambda$ is, the smoother the resulting distribution is. This is shown in Figure \ref{fig:rml_lambda_effect}. It is also important to notice that, even though an increment in $\lambda$ of $1 \times 10^{-3}$ can be considered as insignificant to the resulting disribution, this is not the case. In order to understand the effect of lambda to the resulting distribution, this parameter can be understood as the window size, formulated as $n_{\lambda} = \frac{1}{1-\lambda}$, meaning the number of previous observations considered at each time period $t \in T$. As a result, between Figures \ref{fig:0999} and \ref{fig:09999}, there is difference of one order of magnitude in terms of previous observations considered. That means that, while Figure \ref{fig:0999} shows a resulting distribution considering the previous $1 \times 10^{3}$ observations, Figure \ref{fig:09999} calculates the resulting distribution based on the $1 \times 10^{4}$ previous observations. 

\begin{figure}[ht!]
\centering     %%% not \center
\subfigure[Density function with $\lambda =0.999$]{\label{fig:0999}\includegraphics[width=62mm]{ChapterAggFlexForecast/Figures/final_density_RML_lambda_0999_2.png}}
\subfigure[Density function with $\lambda =0.9999$]{\label{fig:09999}\includegraphics[width=62mm]{ChapterAggFlexForecast/Figures/final_density_RML_lambda_09999_2.png}}
\caption{Effect of $\lambda$ in the resulting probability density function}
\label{fig:rml_lambda_effect}
\end{figure}



\subsubsection{Single Model}\label{Sect:ResultsLevel2SingleModel} 

Under the second level of the hierarchy, the data used are those that ensure that flexibility is available, meaning that the first condition of the model is met. In this case, a single model is created at the beginning of the case study which is being updated at each time period of the time horizon considered under study. The dataset has been split into train and test, using the last three months as a validation of the model. The Grid-Search method has been used to determine the optimal value of the hyper-parameter $\lambda$, being the one that results in the minimum value of the LS.  Based on the results obtained in the training phase, the optimal value of lambda, $\lambda^{*}$, has been $0.997$. This parameter has then been introduced as a fixed parameter under the test set, in order to validate the model and test that the resulting model is not overfitted. Table \ref{tab:train-test-score} shows the resulting scores under the train and validation sets. This optimal forgetting factor of $\lambda^{*} = 0.997$ informs that the equivalent window size for estimating the conditional flexibility value at $t+1$ is equal to 333.33 time periods, in this case, hours. This value is significant to show that implementing recursive maximum likelihood estimation methods allows the forgetting factor to consider enough previous time periods to adapt to slow variations in the time horizon of study. On the contrary, this value could imply that it cannot compensate extreme flexibility values in specific time periods. However, in this case the performance of the LS score does not show any sign of it, and hence the consideration of a greater number of previous observations results in a better likelihood score. 
Figures \ref{fig:conditional_density_16_09} and \ref{fig:conditional_density_31_12} show the resulting flexibility value distribution for a given time period. These figures reveal the probability for a specific flexibility value in kWh under a specific period of time. One can see how at each time period the resulting distribution function evolves, according to the recursive formulation presented in Section \ref{Sect:FlexModeling}. The resulting conditional flexibility shows that there are two peaks where the probability is higher, compared to the other values, being around 0.53~kWh and 3.51 kWh. 


\begin{table}[htbp]
\centering
\caption{Results of the cross validation procedure for hyperparameter definition for the Single Model in the second level of the hierarchy, using the log-likelihood score (LS) as a performance score.}
\vspace*{3mm}
\label{tab:train-test-score}
\resizebox{0.5\textwidth}{!}{%
\begin{tabular}{ccc} 
\toprule
                                                                                        & \textbf{Train set} & \textbf{Validation set}  \\ 
\hline
\begin{tabular}[c]{@{}c@{}}\textbf{LS }\\\textbf{ ($\lambda^{*}$ = 0.997)}\end{tabular} & 1.87               & 2.04                     \\
\bottomrule
\end{tabular}
}
\end{table}

%\begin{figure}[]
%     \centering
%     \begin{subfigure}[]{}
%         \centering
%         \includegraphics[width=\textwidth]{ChapterAggFlexForecast/Figures/finaldensity_corrected_1_20180916_1700_v2.png}
%         \caption{16/09/2018 17:00}
%         \label{fig:conditional_density_16_09}
%     \end{subfigure}
%     \begin{subfigure}[]{}
%         \centering
%         \includegraphics[width=\textwidth]{ChapterAggFlexForecast/Figures/finaldensity_corrected_1_20181231_23_v2.png}
%         \caption{31/12/2018 23:00}
%         \label{fig:conditional_density_31_12}
%     \end{subfigure}
%    \caption{Single Model - Flexibility Value conditional densities at specific time periods}
%    \label{fig:conditional_densities_L1}
%\end{figure}



\begin{figure}[htbp]
\centering     %%% not \center
\subfigure[16/09/2018 17:00]{\label{fig:conditional_density_16_09}\includegraphics[width=62mm]{ChapterAggFlexForecast/Figures/finaldensity_corrected_1_20180916_1700_v2.png}}
\subfigure[31/12/2018 23:00]{\label{fig:conditional_density_31_12}\includegraphics[width=62mm]{ChapterAggFlexForecast/Figures/finaldensity_corrected_1_20181231_23_v2.png}}
\caption{Single Model - Flexibility Value conditional densities at specific time periods}
\label{fig:conditional_densities_L1}
\end{figure}


\subsubsection{Hourly Model}  \label{Sect:ResultsLevel224hModel}

One can use the same approach for estimating flexibility, creating a model for each time period, in this case for every hour, instead of using a single model that is updated at each time period, as developed in the previous section. Then, this model could show differences in the density function for each time period of the day, since it is updated every day at that given time when a new observation is introduced into the model. According to the mathematical formulation, there is only a single forgetting factor $\lambda$ to be considered in each model. However, this approach considers 24 models, one for each hour of the day. To assess the possibility of considering a single $\lambda$ forgetting factor for all models, Grid-Search technique is implemented, considering a wide range of forgetting factors. The first set of analyses confirmed that the behaviour of the forgetting factor is the same for each hourly model, as seen in Figure \ref{fig:L2_M2_LS}. Interestingly, this analysis revealed that, in some hours of the day, the forgetting factor plays a key-role in the resulting score. This can be seen for example in the early morning between 4:00 and 7:00, whereas in the afternoon there is not such influence in the resulting overall score.  

\begin{figure}[htbp]
\centerline{\includegraphics[width=0.8\columnwidth]{ChapterAggFlexForecast/Figures/L2M2_LS_v3.png}}
\caption{Training set scores for each time period using different $\lambda$ forgetting factors}
\label{fig:L2_M2_LS}
\end{figure}

The model presented here shows slightly different results, in terms of the value of the forgetting factor $\lambda$ as well as the resulting density functions. The minimum LS obtained under the train set provides the optimal forgetting factor of $\lambda^* = 0.96$, being the LS score of 1.75 in this case (Table \ref{tab:train-test-score-L2M2}). 

%\begin{table}[]
%\centering
%\caption{Results of the cross validation procedure for hyperparameter definition for the Hourly Model in the second level of the hierarchy, using LS as a performance score.}
%\label{tab:train-test-score-L2M2}
%\resizebox{0.5\textwidth}{!}{%
%\begin{tabular}{c|c|c|}
%\cline{2-3}                                             
%& \textbf{Train set} & \textbf{Validation set} \\ \hline
%\multicolumn{1}{|c|}{\textbf{\begin{tabular}[c]{@{}c@{}}LS\\ ($\lambda^{*}$ = 0.96)\end{tabular}}} & 1.75          &   2.00                \\ \hline
%\end{tabular}%
%}
%\end{table}



\begin{table}[htbp]
\centering
\caption{Results of the cross validation procedure for hyperparameter definition for the Hourly Model in the second level of the hierarchy, using LS as a performance score.}
\label{tab:train-test-score-L2M2}
\vspace*{3mm}
\resizebox{0.5\textwidth}{!}{%
\begin{tabular}{ccc} 
\toprule
                                                                                        & \textbf{Train set} & \textbf{Validation set}  \\ 
\hline
\begin{tabular}[c]{@{}c@{}}\textbf{LS }\\\textbf{ ($\lambda^{*}$ = 0.96)}\end{tabular}  & 1.75          &   2.00 \\
\bottomrule
\end{tabular}
}
\end{table}

In order to validate the model, this parameter is fixed under the validation set, and the LS score is calculated again, being in this case of 2.00. One could argue that the forgetting factor value $\lambda^{*}$ could be too low for a forgetting factor, since it will overwrite previous density distribution faster than values closer to 1. However, this approach provides a density function for each hour of a day, and hence it is updated daily when new data are fed into the model. An optimal forgetting factor of $\lambda^* = 0.96$ considers an effective number of observation of 25 data points, being the equivalent of 25 days in memory to compute the following day. Figures \ref{fig:L2-M2_20180630} and \ref{fig:L2-M2_20181231} show a significant difference in the resulting density function for a given time of the day and a given day, being in both cases a wider range of flexibility available around 17:00 than at 00:00. 


%\begin{figure}[!h]
%     \centering
%     \begin{subfigure}[b]{}
%         \centering
%         \includegraphics[width=\textwidth]{ChapterAggFlexForecast/Figures/L2_M2_densities_00-12-17_20180630_lambda096.png}
%         \caption{30/06/2018}
%         \label{fig:L2-M2_20180630}
%     \end{subfigure}
%     \begin{subfigure}[b]{}
%         \centering
%         \includegraphics[width=\textwidth]{ChapterAggFlexForecast/Figures/L2_M2_densities_00-12-17_20181231_lambda096.png}
%         \caption{31/12/2018}
%         \label{fig:L2-M2_20181231}
%     \end{subfigure}
%    \caption{Hourly Model - Flexibility value conditional densities at specific time periods}
%    \label{fig:conditional_densities_L2}
%\end{figure}


\begin{figure}[htbp]
\centering    
\subfigure[30/06/2018]{\label{fig:L2-M2_20180630}\includegraphics[width=62mm]{ChapterAggFlexForecast/Figures/L2_M2_densities_00-12-17_20180630_lambda096.png}}
\subfigure[31/12/2018]{\label{fig:L2-M2_20181231}\includegraphics[width=62mm]{ChapterAggFlexForecast/Figures/L2_M2_densities_00-12-17_20181231_lambda096.png}}
\caption{Hourly Model - Flexibility value conditional densities at specific time periods}
\label{fig:conditional_densities_L2}
\end{figure}


The initial hypotheses that the hourly model would perform better in terms of summarized performance is validated. However, when comparing the two different approaches, the Hourly Model obtains only slightly better results. The performance score (LS) is 6.85\% lower under the train set. Besides, when evaluating the model under the validation set under the optimal value of the forgetting factor, the score is almost equal for both approaches, obtaining an improvement of 2\%. In terms of computational resources, both models are fast. The Single Model performs approximately 29\% faster than the model containing 24 hourly density functions. The reason being is that the latter needs more parameters and spends more resources on storing the density functions, derivatives and hessian values for each model.  It is plausible that, in both models, the estimated densities show greater roughness than the expected by using KDE approaches, which could be improved as a further research by means of regularization or roughness penalization applied to splines (Figures \ref{fig:conditional_density_16_09}, \ref{fig:conditional_density_31_12}, \ref{fig:L2-M2_20180630}, \ref{fig:L2-M2_20181231}).
As a conclusion for the calculation of the conditional density, the choice between the Single Model and the Hourly Model can be made according to the final objective of the conditional density. In cases where the time period at which flexibility could be provided is known, for example, the possibility of EV fleet portfolios, a model for each period could offer better results than a single one updated throughout all time periods. 

%\begin{table}[]
%\centering
%\caption{Results overview between the two models developed under level~2 of the hierarchical model for conditional flexibility estimation.}
%\label{tab:level2-scores}
%\resizebox{1\textwidth}{!}{%
%\begin{tabular}{c|c|c|c|}
%\cline{2-4}
% &
%  \textbf{\begin{tabular}[c]{@{}l@{}}LS train {[}-{]}\end{tabular}} &
%  \textbf{\begin{tabular}[c]{@{}l@{}}LS validation {[}-{]}\end{tabular}} &
%  \textbf{\begin{tabular}[c]{@{}l@{}}Processing Time {[}s{]}\end{tabular}} \\ \hline
%\multicolumn{1}{|l|}{\textbf{Level 2 - Single Model}} &
%  1.87 &
%  2.04 &
%  0.97
%   \\ \hline
%\multicolumn{1}{|l|}{\textbf{Level 2 - Hourly Model}} &
%  1.75 &
%  2.00 &
%  1.36
%   \\ \hline
%\end{tabular}%
%}
%\end{table}



\begin{table}[htbp]
\centering
\caption{Results overview between the two models developed under level~2 of the hierarchical model for conditional flexibility estimation.}
\vspace*{3mm}
\label{tab:level2-scores}
\resizebox{0.9\textwidth}{!}{%
\begin{tabular}{lccc} 
\toprule
\multicolumn{1}{c}{}            & \textbf{LS train [-]} & \textbf{LS validation [-]} & \textbf{Processing Time [s]}  \\ 
\hline
\textbf{Level 2 - Single Model} & 1.87                  & 2.04                       & 0.97                          \\
\textbf{Level 2 - Hourly Model} & 1.75                  & 2.00                       & 1.36                          \\
\bottomrule
\end{tabular}
}
\end{table}

\subsection{Aggregated flexibility forecast: Combination of level 1 and level 2}  \label{Sect:ResultsFinalOutcome} 
The final outcome of this methodology is the expected flexibility value as a combination of the two-level hierarchical model, as follows

\begin{equation}
  \mathbb{E}[Y] = \mathbb{E}[Y|X=1] \,  P[X=1]
\end{equation}

where the final value, meaning the flexibility forecast is represented as the probability of the flexibility to be available, $ P[X=1]$,  times the expected value of the conditional probability, represented as $\mathbb{E}[Y|X=1]$. Figures \ref{fig:final_outcome_M1} and \ref{fig:final_outcome_M2} illustrate the final time-series flexibility forecast for both the Single Model and the Hourly Model, compared against the real measurement over an arbitrary week of the case study.

%\begin{figure}[]
%     \centering
%     \begin{subfigure}[]{}
%         \centering
%         \includegraphics[width=\textwidth]{ChapterAggFlexForecast/Figures/final_forecast_M1_final_v2.png}
%         \caption{Single Model}
%         \label{fig:final_outcome_M1}
%     \end{subfigure}
%     \begin{subfigure}[]{}
%         \centering
%         \includegraphics[width=\textwidth]{ChapterAggFlexForecast/Figures/final_forecast_M2_final.png}
%         \caption{Hourly Model}
%         \label{fig:final_outcome_M2}
%     \end{subfigure}
%    \caption{Time-series of flexibility consumption under the case study, considering both measurements and predictions, over an arbitrarily chosen episode of one week.}
%    \label{fig:conditional_densities_2}
%\end{figure}


\begin{figure}[htbp]
\centering    
\subfigure[30/06/2018]{\label{fig:final_outcome_M1}\includegraphics[width=62mm]{ChapterAggFlexForecast/Figures/final_forecast_M1_final_v2.png}}
\subfigure[31/12/2018]{\label{fig:final_outcome_M2}\includegraphics[width=62mm]{ChapterAggFlexForecast/Figures/final_forecast_M2_final.png}}
\caption{Hourly Model - Flexibility value conditional densities at specific time periods}
\label{fig:conditional_densities_2}
\end{figure}


The combination of Level 1 and Level 2 using the Single Model based on a single density function updated at each time period $t \in T$, results in a RMSE value of 1.82 kWh. In the case of using the Hourly Model based on 24 density functions for each hour of the day, the final RMSE obtained is 1.68 kWh. An overview of the RMSE and MAE scores obtained for the flexibility final outcome is shown in Table \ref{tab:final_scores}. Results prove that, in both cases, this approach performs significantly better against a benchmark modeled with a SES approach which formulation is outlined in \cite{Hyndman2021}, as shown in Tables \ref{tab:final_scores_improv_M1} and  \ref{tab:final_scores_improv_M2}. 
Besides, the final flexibility outcome provides a better performance when combining both levels of the hierarchy by means of the Hourly Model, assessing its performance with the RMSE and MAE scores. 

%\begin{table}[]
%\centering
%\caption{Results of the final flexibility forecast outcome combining the two levels of the hierarchy. Results are for SES, and RML KDE for the Single Model and the Hourly Model.}
%\label{tab:final_scores}
%\resizebox{0.6\textwidth}{!}{%
%\begin{tabular}{l|l|l|}
%\cline{2-3}
% & \textbf{\begin{tabular}[c]{@{}l@{}}RMSE {[}kWh{]}\end{tabular}} & \textbf{\begin{tabular}[c]{@{}l@{}}MAE {[}kWh{]}\end{tabular}} \\ \hline
%\multicolumn{1}{|l|}{\textbf{SES Benchmark}}   & 2.83 & 2.04 \\ \hline
%\multicolumn{1}{|l|}{\textbf{Single Model}} & 1.82 & 0.99 \\ \hline
%\multicolumn{1}{|l|}{\textbf{Hourly Model}}    & 1.68 & 0.92  \\ \hline
%\end{tabular}%
%}
%\end{table}
 
\begin{table}[htbp]
\centering
\caption{Results of the final flexibility forecast outcome combining the two levels of the hierarchy. Results are for SES, and RML KDE for the Single Model and the Hourly Model.}
\vspace*{3mm}
\resizebox{0.6\textwidth}{!}{%
\label{tab:final_scores}
\begin{tabular}{lll} 
\toprule
                       & \textbf{RMSE [kWh]} & \textbf{MAE [kWh]}  \\ 
\hline
\textbf{SES Benchmark} & 2.83                & 2.04                \\
\textbf{Single Model}  & 1.82                & 0.99                \\
\textbf{Hourly Model}  & 1.68                & 0.92                \\
\bottomrule
\end{tabular}
}
\end{table} 
 
 
%\begin{table}[]
%\centering
%\caption{Performance overview for the final flexibility outcome using Level 2 - Single Model as conditional flexibility, against the benchmark. Performance is evaluated with RMSE and MAE criteria.}
%\label{tab:final_scores_improv_M1}
%\resizebox{0.95\textwidth}{!}{%
%\begin{tabular}{c|c|c|c|}
%\cline{2-4}
%                                              & \textbf{SES Benchmark} & \textbf{Level 1 / Single Model} & \textbf{Improvement {[}\%{]}} \\ \hline
%\multicolumn{1}{|l|}{\textbf{RMSE {[}kWh{]}}} & 2.83                     & 1.82                       & 35.69                         \\ \hline
%\multicolumn{1}{|l|}{\textbf{MAE {[}kWh{]}}}  & 2.04                     & 0.99                       & 51.47                         \\ \hline
%\end{tabular}%
%}
%\end{table}

\begin{table}[htbp]
\centering
\caption{Performance overview for the final flexibility outcome using Level 2 - Single Model as conditional flexibility, against the benchmark. Performance is evaluated with RMSE and MAE criteria.}
\vspace*{3mm}
\label{tab:final_scores_improv_M1}
\resizebox{0.95\textwidth}{!}{%
\begin{tabular}{lccc} 
\toprule
\multicolumn{1}{c}{} & \textbf{SES Benchmark} & \textbf{Level 1 / Single Model} & \textbf{Improvement [\%]}  \\ 
\hline
\textbf{RMSE [kWh]}  & 2.83                   & 1.82                            & 35.69                      \\
\textbf{MAE [kWh]}   & 2.04                   & 0.99                            & 51.47                      \\
\bottomrule
\end{tabular}
}
\end{table}

%\begin{table}[]
%\centering
%\caption{Performance overview for the final flexibility outcome using Level 2 - Hourly Model as conditional flexibility, against the benchmark. Performance is evaluated with RMSE and MAE criteria.}
%\label{tab:final_scores_improv_M2}
%\resizebox{0.95\textwidth}{!}{%
%\begin{tabular}{c|c|c|c|}
%\cline{2-4}
%                                              & \textbf{SES Benchmark} & \textbf{Level 1 / Hourly Model} & \textbf{Improvement {[}\%{]}} \\ \hline
%\multicolumn{1}{|l|}{\textbf{RMSE {[}kWh{]}}} & 2.83                     & 1.68                       & 40.63                         \\ \hline
%\multicolumn{1}{|l|}{\textbf{MAE {[}kWh{]}}}  & 2.04                     & 0.92                       & 54.90                         \\ \hline
%\end{tabular}%
%}
%\end{table}

\begin{table}[htbp]
\centering
\caption{Performance overview for the final flexibility outcome using Level 2 - Hourly Model as conditional flexibility, against the benchmark. Performance is evaluated with RMSE and MAE criteria.}
\vspace*{3mm}
\label{tab:final_scores_improv_M2}
\resizebox{0.95\textwidth}{!}{%
\begin{tabular}{lccc} 
\toprule
\multicolumn{1}{c}{} & \textbf{SES Benchmark} & \textbf{Level 1 /  Hourly Model} & \textbf{Improvement [\%]}  \\ 
\hline
\textbf{RMSE [kWh]}  & 2.83                     & 1.68                       & 40.63                      \\
\textbf{MAE [kWh]}   & 2.04                     & 0.92                       & 54.90                      \\
\bottomrule
\end{tabular}
}
\end{table}

\section{Chapter remarks} \label{Sect:Conclusions}
This paper proposes a new aggregated flexibility forecast model based on a hierarchical formulation and an online adaptive bandwidth Kernel Density Estimation approach based on Recursive Maximum Likelihood. Probabilistic densities of flexibility estimation have been explicitly formulated under this approach. Results based on a real dataset case study show that it is possible to approximate and track an unknown distribution under an online framework. This approach provides a fast tool for obtaining a probabilistic forecast of the flexibility availability and its value. Furthermore, by implementing a probabilistic forecast, one can manage the uncertainty given by the nature of the demand-side flexible assets. 
It quantifies the flexibility available within the portfolio without the need of creating a specific model for each asset-type, while at the same time avoiding the storage of a large amount of user data, which is sometimes difficult to obtain due to data privacy or third-parties contracts. 
This model estimates the aggregated available flexibility with low computation resources and input data compared to individual approaches as HEMS optimization in terms of computational burdens. It uses aggregated data, and the number of users or assets does not affect the flexibility forecast algorithm computation time. However, when this approach for estimating flexibility is to be implemented, it will require scheduling the specific assets needed to activate this flexibility, allowing this to be done at a later stage and using optimization techniques at household level. To further this research, the implementation of this novel formulation should be based not only on flexible consumption but also considering generation from PV and prosumption from ESS, eventually increasing the aggregated flexibility that can be provided. Datasets including different and a larger amount of flexible assets could be helpful to replicate and scale up the approach presented in this paper. Further approaches such as regularization or roughness penalization applied to splines could be included in the model.

