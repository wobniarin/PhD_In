\chapter{Flexibility Services}
\label{chapterFlexibility}
\chaptermark{Flexibility Services for Distribution Networks}

\section{Introduction}

Also, the penetration of DG into the MV and
LV grid will suppose some challenges, as seen in Section 3.3, which will need to be addressed
by the DSOs via active grid management. Finally, the provision of local flexibility will help not
only to secure the grid operation but also to improve grid efficiency efficiency during normal
operation time [18].

For these reasons, improved flexibility markets are being recognized in the e-Directive as a
pillar to support the safer and more efficient use of the existing grids, and to enhance the HC of
distribution feeders. Since the scope of this work is to research those regulations that enhance
RESs penetration while guaranteeing safe operation of the power grid, it is interesting to study
how flexibility markets could be designed in order to promote DERs participation.

\begin{figure}[]
	\centering
	\includegraphics[width=0.7\columnwidth ]{ChapterFlexServices/Figures/phd_intro_ii.pdf}
	   %\vspace*{-8cm}
		\caption{Chapter objective based on the PhD scope}
	\label{fig:chapter_obj_ii}  
\end{figure}

\section{The importance of flexibility}

In the traditional organization of the power grid, large PGMs were forced to be able to provide
flexibility. Some of these requirements for large PGMs are still considered in the new RfG NC.
However, most of the requisites are not mandatory for smaller PGMs, and in some MSs legislation,
RES power parks that fit in the large PGMs definition are excluded from the fulfillment
of these requirements. With the new paradigm decreasing the number of synchronous PMGs
and more difficulties than ever to forecast demand and generation, the large PGMs remaining
may not be able to handle the flexibility required to keep the grid working correctly. This will
suppose an increased need for flexibility [35]. Also, the penetration of DG into the MV and
LV grid will suppose some challenges, as seen in Section 3.3, which will need to be addressed
by the DSOs via active grid management. Finally, the provision of local flexibility will help not
only to secure the grid operation but also to improve grid efficiency efficiency during normal
operation time [18].
For these reasons, improved flexibility markets are being recognized in the e-Directive as a
pillar to support the safer and more efficient use of the existing grids, and to enhance the HC of
distribution feeders. Since the scope of this work is to research those regulations that enhance
RESs penetration while guaranteeing safe operation of the power grid, it is interesting to study
how flexibility markets could be designed in order to promote DERs participation.


\section{Flexibility definition}
The electricity system has one intrinsic flaw; the generation-consumption link which, aside from
ESSs, is not breakable and in the DG new paradigm supposes a big challenge for the grid operators
in terms of system safety. From a time-perspective this problematic has two sides:
\begin{itemize}
\item Long-term reliability (Capacity adequacy): Defined in [68] as "the ability of the electric
system to supply the aggregated electrical demand and energy requirements of costumers at all
times."
\item Short-term reliability (Flexibility): Defined in [68] as:"the ability of the electric system to
withstand sudden disturbances."
\end{itemize}

After this generic definition of Flexibility, more partial approaches showing the value of flexibility
for diverse grid stakeholders, extracted from literature are given:
\begin{itemize}
\item Consumer approach: The Office of Gas and Electricity Markets (Ofgem) of UK defines flexibility
[69] as "modifying generation and/or consumption patterns in reaction to an external signal
(such as a change in price) to provide a service within the energy system." Furthermore, they
define as newflexibility methods Demand-Side Management (DSM), Energy Storage and
Distributed Generation.
\item Transmission system approach: One definition of flexibility given by the EU [70] is: "the
capability of the power system to cope with the short/mid-term variability of generation (like renewable
energy) and demand so that the system is kept in balance."
The Universal Smart Energy Framework (USEF) points out that TSOs can benefit from flexibility
services to cope with different problematic: from ancillary services for balancing purposes to constraint management and adequacy services [71].
\item Distribution system approach: The new DG paradigm creates the need of a new approach
to flexibility from the distribution part of the grid. Using as reference the definition
of grid-oriented services given in [25], distribution system flexibility can be defined
as the capability of the distribution system to cope with locational short-term congestion
of feeders and also for distribution grid balancing purposes. Furthermore, as USEF points
out in the report Flexibility Value Chain [71] , it also can be used to increase performance
and efficiency by using demand-side flexibility which helps defer or avoid the costs of grid
reinforcements
\item So, it is inherent to all the perspectives seen that flexibility is something that provides margin to
the grid to maintain instantaneous stable and safe operation, and in some cases during normal
operation periods it can improve the way the grid is working.
\end{itemize}
\subsection{Flexibility provision}
One way to approach the power system is by dividing it into generation and consumption, two
antagonist concepts that are nowadays merging due to DERs and ESSs. Both sides can provide
flexibility: Generation-side flexibility and demand-side flexibility. 

The new DG and Smart Grid paradigm turn the spotlight towards the DR concept. Hitherto, it
was only provided by large loads with the ability to modify their consumption habits. From now
on, DRwill be enhanced by the technological advances and the introduction ofDGatMVand LV
level. In this newparadigm, even prosumers will be able to adapt their load profiles towards the
grid. However, the Generation-Side Flexibility will still be a key agent on the flexibility markets. Demand-side Management can be approached from two perspectives defined in [72]:
\begin{itemize}
\item Explicit Demand-Side Flexibility: "Dispatchable flexibility that can be traded (similar to generation
flexibility) on the different energy markets (wholesale, balancing, system support and reserves
markets). This is usually facilitated and managed by an aggregator that may be an independent
service provider or a supplier."
\item Implicit Demand-Side Flexibility: "Consumer's reaction to price signals. Where consumers
can choose hourly or shorter-term market pricing, reflecting variability on the market and the network,
they can adapt their behavior to save on energy expenses. This type of Demand-Side Flexibility
(DSF) is often referred to as "price-based" DSF."
\end{itemize}

While both kinds of DR are considered in the new European framework, Explicit Demand-Side
flexibility, together with Generation-side flexibility, are the ones towards the EU is legislating.
This is mostly because of their product nature that makes them market sellable, which supposes
a step forward on the predictions of capacity balancing of future power grids. At the same time,
if consumers can provide services to the grid operators, this will suppose empowerment for
them and possibly a push for the widespread of small RES installations.

\subsection{Flexibility end-users}
\subsubsection{market oriented}
\subsubsection{System oriented}

\subsection{EU Guidelines and Legislation on Flexibility}
There are several approaches to add flexibility to the grid; the following list enumerates them and evaluates them from the prosumer perspective: 
\begin{enumerate}
\item Compulsory provision: Technical and operational requirements for all the generators and
loads is the traditional approach before the creation of the European balancing markets.
It is still a thing today on some legislations, but mainly for large PGMs. Imposing these requirements
to the smaller generators/loads nowadays seems technically impossible due to
the impossibility to control and monitor all the assets, plus it may be unfair for prosumers,
and could collide with their interests.
\item Bilateral contracts: TSO agrees with some capacity provider on an over-the-counter contract
to acquire capacity provision. These kinds of contracts are long-term ones, and the
capacity provided is well over anything a prosumer can provide. It is the least transparent
way to provide flexibility, but it can be a way to provide safety to some significant
investments focused on earning money from energy/capacity provision.
\item Flexibility provision by TSO or DSO: DSO and TSO as responsible for the grid management
may seem to be one of the prominent agents interested in flexibility provision.
However, due to the objectives of market liberalization and unbundling of the power grid
settled by the EU, DSOs and TSOs shall not be allowed to own either PGMs or ESSs (other
than justified exemptions; see section 4.3.1). Summarizing, this leads to the impossibility
of the system operators to provide such services
\item Flexiblity Markets: Since the publication of the First Energy Package, the creation of an
European internal electricity market has been the main objective. From this perspective,
nowadays the EU is promoting the use of flexibility markets as the primary capacity mechanism
(e-Regulation Art. 22), and also the creation of a standardized portfolio of products
to enhance the transnational exchange of capacity. The main argument to discourage other
options is that Europe as a whole is nowadays in over-capacity, and traditional capacity
mechanisms tend to be highly inefficient [19, 27, 35].
\end{enumerate}

The Third Energy Package follows the path established by the EU in terms of the creation of
an internal European market, promoting the unbundling of the electric system structures and therefore opening the system to private investors. Then, there was the Energy Efficiency Directive
(2012/27/EC), which is the first one to look forward to the use of "leveled-for-all-users"
energy flexibility markets as the primary agents for the transformation to a more efficient energy
system on the Article 18. Lately, the publication of the Electricity Balancing Guidelines (EB
GL; 2017/2195) has been an enormous step forward in terms of standardization of balancing
products and guidelines for MSs to establish their own balancing markets. Finally, the publication
of the CEP outcomes is a new boost for flexibility markets, amending problematic not
treated in previous directives and facing new challenges. 
However, it is important to consider that up until the CEP publication, when Europe was talking
about flexibility markets it was focused on ancillary services related to frequency provision.
This kind of product aims to balance generation and demand so TSOs centrally operate this market.
Recently in the e-Directive (Art.59), the need for network codes related to non-frequency
ancillary services is stated for the first time. This will suppose the opening of a new, but also
unexplored, decentralized market for congestion management at DSO level.

\section{Mathematical formulation for flexibility definition}
incloure aqui idea review hussain sobre flexibility review 


\section{Flexibility forecast: Introduction to time series}


\section{Discussion}
\section{Conclusion}

	


