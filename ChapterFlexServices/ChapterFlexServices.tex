\chapter{Framework definition and mathematical formulation of flexibility}
%\chapter{Flexibility Services: Defining the framework}
\label{chapterFlexibility}
\chaptermark{Framework definition and mathematical formulation of flexibility}

\section{Objectives and contributions}
One of the outcomes of the previous chapter is that flexibility can be a key enabler of the energy transition, due to the possibility to create a new service for network operators, aggregators, retailers and end-users. Even though in the first years of the discussion on local markets energy was the main product to be provided, reviewed literature showed an increasing interest on enabling flexibility. However, there is still some discussion on how flexibility can be defined, modeled, forecast and priced. This chapter focuses on the definition of flexibility according to objective $(ii)$ of the PhD research, and provides a formulation which will be later used for providing this service to the distribution system operator in Chapter 4. The relationship between the thesis objective within the whole ecosystem of the research is shown in Figure \ref{fig:chapter_obj_ii}.


\begin{figure}[htbp]
	\centering
	\includegraphics[width=0.7\columnwidth ]{ChapterFlexServices/Figures/phd_intro_ii.pdf}
	   %\vspace*{-8cm}
		\caption{Chapter objective based on the PhD scope}
	\label{fig:chapter_obj_ii}  
\end{figure}

The previous chapter highlighted the role of active grid management implemented by DSOs in order to increase the hosting capacity (HC) of distribution networks, and that flexibility could be one of the main services to achieve it. Furthermore, the provision of local flexibility will help not only to secure the grid operation but also to improve grid efficiency during normal operation time [18]. For these reasons, improved flexibility markets are being recognized in the e-Directive as a pillar to support the safer and more efficient use of the existing grids, and to enhance the HC of distribution feeders. Since the scope of this work is to research those regulations that enhance RESs penetration while guaranteeing safe operation of the power grid, it is interesting to study how flexibility markets could be designed in order to promote DERs participation.


\section{Flexibility definition}
The electricity system has one intrinsic flaw; the generation-consumption link, which generally is not breakable. This flaw supposes a big challenge for grid operators in terms of system safety in the energy transition roadmap. From a time-perspective this problem has two sides, according to \cite{FINON2008143}:

\begin{enumerate}
\item \textbf{Long-term reliability (Capacity adequacy)}: The ability of the electric system to supply the aggregated electrical demand and energy requirements of costumers at all times.  \cite{FINON2008143}
\item \textbf{Short-term reliability (Flexibility)}: The ability of the electric system to withstand sudden disturbances.
\end{enumerate}

This can be considered as the initial and most generic definition of flexibility. However, in the recent years more specific definitions of flexibility have been provided, based on the final client using it and the agent providing it. These definitions are given below:

\begin{enumerate}
\item \textbf{Consumer approach}: From the consumer point of view, flexibility is meant to be the modification of generation and consumption patterns, employing DSM, in reaction to an external signal such as a change in price, to provide a service within the energy system \cite{OfficeofGasandElectricityMarketsOfgem2015}. They also include as new flexibility methods energy storage and distributed generation. 
\item \textbf{Transmission system approach}: From the Transmission System Operator (TSO) perspective, it is understood as the capability of the power system to cope with the short and mid-term variability of renewable generation and demand so that the system is kept in balance \cite{Profumo2016, EuropeanNetworkofTransmissionSystemOperatorsforElectricityENTSO-E2015}.
The Universal Smart Energy Framework (USEF) points out that TSOs can benefit from flexibility services to cope with different problematic: from ancillary services (AS) for balancing purposes to constraint management and adequacy services \cite{USEF2018}.
\item \textbf{Distribution system approach}: Lastly, flexibility services for the DSO are related to the capability of the distribution network to cope with located short-term congestion of feeders, and also for distribution grid balancing purposes \cite{Minniti_2018, Khatami2018}. 
\end{enumerate}
It is inherent to all the perspectives seen that flexibility is something that provides margin to the grid to maintain instantaneous stable and safe operation, and in some cases during normal operation periods it can improve the way the grid is working. In the case of the study in this research, flexibility is to be provided to the distribution network operator for operation purposes. Consequently, this flexibility bought by the DSO is provided by the demand-side or consumer side, but the main objective of this flexibility is to increase the grid hosting capacity and enhace active grid management at distribution level. However, it is important to highlight the main applications and benefits for the power system agents by activating flexibility. 

\begin{enumerate}
\item \textbf{Prosumer/consumer approach}: Self-balancing in terms of maximization of power used coming from installed DERs, time of use optimization based on load shifting and peak-shaving that leads to a reduction of energy costs \cite{Olivella-Rosell2018}.
\item \textbf{Transmission system approach}: Ancillary services for balancing purposes, constraint management and adequacy services \cite{USEF2018}.
\item \textbf{Distribution system approach}: Congestion management, voltage control, avoidance of grid reinforcements and associated costs \cite{USEF2018, Olivella-Rosell2018}.
\item \textbf{Balance Responsible Party/Retailer approach}: Portfolio optimization, imbalances and penalties minimization \cite{Olivella-Rosell2018}.
\end{enumerate}

In this case, the main objective is to provide a flexibility service to the DSO. By doing so, the DSO can implement active grid management, increasing the network hosting capacity and avoiding the grid reinforcement. At the same time, and since this flexibility will be provided by the demand-side or end-users, this will also provide the demand-side with the benefits of providing flexibility listed above. 

\section{EU regulation and directives for flexibility services}
There are several approaches to add flexibility to the grid; but this research foscuses on the demand-side flexibility. Hence, the EU legislation for demand-side flexibility is summarized and listed below. 

\begin{enumerate}
\item \textbf{Compulsory provision:} Technical and operational requirements for all the generators and loads is the traditional approach before the creation of the European balancing markets. It is still a thing today on some legislations, but mainly for large generator units. Imposing these requirements to the smaller generators/loads nowadays seems technically impossible due to the impossibility to control and monitor all the assets, plus it may be unfair for prosumers, and could collide with their interests.
\item \textbf{Bilateral contracts:} TSO agrees with some capacity provider on an over-the-counter contract to acquire capacity provision. These kinds of contracts are long-term ones, and the capacity provided is well over anything a prosumer can provide. It is the least transparent way to provide flexibility, but it can be a way to provide safety to some significant investments focused on earning money from energy/capacity provision. 
\item  \textbf{Flexibility provision by TSO or DSO:} DSO and TSO as responsible for the grid management may seem to be one of the prominent agents interested in flexibility provision. However, due to the objectives of market liberalization and unbundling of the power grid settled by the EU, DSOs and TSOs shall not be allowed to own either generator units or energy storage systems. Summarizing, this leads to the impossibility of the system operators to provide such services and hence determines the creation of the aggregator agent.  
\item  \textbf{Flexiblity Markets:} Since the publication of the First Energy Package, the creation of an European internal electricity market has been the main objective. From this perspective, nowadays the EU is promoting the use of flexibility markets as the primary capacity mechanism \cite{Directive2019943} (e-Regulation Art. 22), and also the creation of a standardized portfolio of products to enhance the transnational exchange of capacity. The main argument to discourage other options is that Europe as a whole is nowadays in over-capacity, and traditional capacity mechanisms tend to be highly inefficient \cite{Hancher2017, validzic2017clean}.
\end{enumerate}

As can be seen from the items listed below, the current EU Guidelines still focus the flexibility provisions based on the same structures thought for TSOs, by means of compulsory provision. However, it is clearly stated that this implies several difficulties when it comes to demand-side flexibility.
The Third Energy Package follows the path established by the EU in terms of the creation of an internal European market, promoting the unbundling of the electric system structures and therefore opening the system to private investors. The Efficiency Directive (2012/27/EC) was published is the first one to promote the concept and use of \textit{leveled-for-all-users} energy flexibility markets as the primary agents for the transformation to a more efficient energy system. Lately, the publication of the Electricity Balancing Guidelines (EB GL; 2017/2195) has been an enormous step forward in terms of standardization of balancing products and guidelines for EU-Member States to establish their own balancing markets. Finally, the publication of the CEP \cite{validzic2017clean} outcomes is a new opportunity for flexibility markets, dealing and highlighting the technical and regulatory problems not treated in previous directives. 
Despite this, up until the CEP publication, when Europe was talking about flexibility markets, it was focused on ancillary services related to frequency provision for TSOs. This kind of product aims to balance generation and demand, so TSOs centrally operate this market. In the e-Directive (Art.59), the need for network codes related to non-frequency ancillary services is stated for the first time. This will suppose the opening of a new but unexplored, decentralized market for congestion management at the DSO level.


\section{Flexibility provision by the Demand-Side} \label{sec:FlexibilityProvision}

One way to approach the power system is by dividing it into generation and consumption, two antagonist concepts that are nowadays merging due to DERs and ESSs. Both sides can provide flexibility: Generation-side flexibility and demand-side flexibility. However, the spotlight is set now on the demand-side, by means of demand-side management activities, that have been covered in Chapter \ref{chapterMarkets}. In this section the aim is to define flexibility not only based on demand-side activities, but also from the system perspective, in order to find a common definition that links the demand-side flexibility with the system operator needs. Demand-side management can be approached from two perspectives defined in \cite{USEF2018}:

\begin{enumerate}
\item \textbf{Explicit demand-side flexibility:} It can be understood as the flexibility that can be traded or dispatched. It can also be defined similarly as generation flexibility) on the different energy markets such as the wholesale day-ahead, intraday, balancing system support and reserves; but also by means of direct control and bilateral contracts. Aggregators are the entities in charge of managing and providing this service, which can be considered an independent service provider only for flexibility or a supplier.
\item \textbf{Implicit demand-side flexibility:} That is based on the consumer's reaction to price signals, defining flexibility as a relationship between consumption and electricity price. In that case, consumers can choose hourly or shorter-term energy market pricing, reflecting variability on the market and the network. As a result, they can adapt their behavior to save on energy expenses. This type of demand-side flexibility is often referred to as "price-based".
\end{enumerate}

While both kinds of DR are considered in the new European framework, Explicit Demand-Side flexibility is the one towards which the EU is legislating. This is primarily because of the product nature that makes it market sellable, which supposes a step forward on the predictions of capacity balancing of future power grids. At the same time, if consumers can provide services to the grid operators, this will suppose empowerment for them and possibly a push for the widespread adoption of small RES installations. 

\section{Mathematical formulation for demand-side flexibility definition for DSOs}

Until now flexibility has been defined by the provider of this flexibility, and the final user of this service. However, based on the final user of this service, flexibility can be formulated under two perspectives: the market-oriented approach and the system-oriented approach. This section aims to provide a framework for determining the best flexibility model approach based on the flexibility provider and the flexibility user. In all cases, flexibility is defined as a time-based and power or energy-based signal. In some cases, it can be defined as a power consumption signal, power generation or power variation. 

\subsection{Market-oriented approach}
From the market-oriented perspective, the most common definition of flexibility is determining operating points of flexibility, as defined in \cite{Olivella-Rosell2018}. In this case, a deterministic value of flexibility for each home energy management system (HEMS) is determined and then aggregated and provided to the local flexibility market with an associated cost at each time period to benefit and optimize the flexible assets of a specific household. This study presents the shortcoming that flexibility cannot be modeled as a deterministic process based on the uncertain nature of the demand-side. Flexibility can also be modeled by specifying an upward and downward flexibility band, as stated in \cite{Soares2017}. This paper focuses the flexibility from DERs located in the demand-side, but considering only generators and not any demand-side management activities. They determine first the operating point of the DER considered to provide flexibility. That operating point at each time period $t$ corresponds to the energy bid cleared in the day-ahead electricity market. Furthemore, flexibility in this case is defined as the difference between the expected forecast and the operating point, determining the upward flexibility that could be traded and provided to the DSO. Similarly, they defined downward flexibility by taking the operating point as the upper limit and the expected forecast as the lower boundary. The resulting boundaries are shown in Figure \ref{fig:envelopes}. However, this definition limits the participation to flexibility services only to those DERs that are large enough to participate in the market, not considering any aggregated flexibility definition under that study. This is similar to the flexibility modeling developed in \cite{Nosair2015}. In this case, a flexibility envelope is defined for DERs, mainly wind and solar power plants, in order to provide a formulation for flexibility to be considered by the generation plants owners and provide flexibility to the system operator. 

\begin{figure}[htbp]
	\centering
	\includegraphics[width=0.9\columnwidth ]{ChapterFlexServices/Figures/up_down_flex.pdf}
	   %\vspace*{-8cm}
		\caption{Flexibility envelopes for upward and downward regulation.}
	\label{fig:envelopes}  
\end{figure}

%Based on \cite{Soares2017}

Other research has focused on determining the available flexibility from a specific type of asset. This is the case, for example, of the flexibility available in combined heat and power systems with thermal energy storage in district heating presented in \cite{nuytten2013flexibility}. This paper determines the maximum flexibility available in power units from different aggregated assets, but being all of them of the same nature. Determining flexibility in an aggregated way is a good approach, since it allows the management of uncertainty by jointly considering a set of assets.  

Another approach for calculating the flexibility from the market perspective is modeling the elasticity between price and demand \cite{Gorria2013}, linking the price with the flexibility activation, represented as a consumption increase or decrease. This is also implemented in \cite{Moret2016}, implementing the so-called and defined in Section \ref{sec:FlexibilityProvision} as Implicit demand-side flexibility. However, modeling flexibility as the elasticity between price and demand requires the participation of a control group in order to determine the elasticity curve between price and consumption, being a barrier in some cases where this is not available. As an example, Figure \ref{fig:elasticity} shows the results of flexibility provision based on price-elasticity flexibility. 

\begin{figure}[htbp]
	\centering
	\includegraphics[width=0.9\columnwidth ]{ChapterFlexServices/Figures/pricesignal.pdf}
	   %\vspace*{-8cm}
		\caption{Flexibility signal based on price-demand elasticity.}
	\label{fig:elasticity}  
\end{figure}

% Based on \cite{Gorria2013}

\subsection{System-oriented approach}
From the system-oriented approach, flexibility has mainly been defined as a multiperiod and time-constrained vector, without an associated price to it, as described in \cite{Pinto2017}. By doing so, the main objective is to determine all the possible trajectories the household consumption can take, in order to provide this flexibility to the system operator. This is an interesting approach since it considers the uncertainty associated to demand, and it is shown in Figure \ref{fig:system}. However, the computational resources spent and time required to compute the flexibility trajectories for each household and then aggregate them for operational purposes can lead to scalability limitations. A similar approach for considering uncertainty associated to DERs is developed in \cite{Bremer2013} and \cite{SONNENSCHEIN2015}. In both cases, they determine a starting operating point and the schedule associated to it for the next time steps. Later, in each time step, the trajectory is modified according to a random factor in order to model the uncertainty associated to these flexible assets. 

\begin{figure}[htbp]
	\centering
	\includegraphics[width=0.9\columnwidth ]{ChapterFlexServices/Figures/trajectories.pdf}
	   %\vspace*{-8cm}
		\caption{Stochastic time-constrained flexibility.}
	\label{fig:system}  
\end{figure}
%Based on \cite{Pinto2017}

Hence, each approach has its benefits and shortcomings. It is not easy to provide a single flexibility modeling approach that works for all flexibility providers, asset types, and flexibility users. There are several differences in each of the cases. In conclusion, and to set a framework for the research developed in this PhD, the objective is to define a flexibility signal that works independently of the asset type. Consequently, flexibility is forecast as an aggregated signal, aggregating a portfolio of users and assets by considering their submetering data from the flexible assets but not considering the nature of each flexible asset. 
The flexibility provider is the aggregator, and hence, the prosumers and end-users represented by it. On the other side, the flexibility user is the distribution network operator, intending to use this flexibility for operation purposes in the short-term horizon. As a result, the approach used in this research is the system-oriented approach, not considering any price or cost to it, and neither a market. With that objective in mind, flexibility is to be a short-term decision-making tool for aggregators to know how much flexibility they have in their portfolio that can be provided to the DSO through a bilateral contract. Furthermore, this aggregated flexibility will be forecast using a probabilistic forecast to consider the uncertainty and randomness associated with demand-side flexibility. 

\section{Service interaction}

In this PhD research the aggregator is responsible for scheduling all flexible assets according to different lower-level objective functions and flexibility contracts between the aggregator and the end-users (consumers or prosumers). Figure \ref{fig:agg-pros-dso} presents the process of flexibility availability calculation and activation in the development of the INVADE H2020 Project and BD4OPEM Projects. 

Firstly, the aggregator collects the historical submetering data of the flexible assets from the different end-users in the portfolio. Once the data has been collected by the aggregator, flexibility can be modeled. In this step, the energy-based collected data is aggregated at each time period, and hence the maximum flexibility from the demand-side is calculated. Under this stage it is assumed that under the contract established between the aggregator and the end-user, direct control of the flexible assets is given. 

Once the flexibility has been modeled, the available flexibility is forecast based on the approach detailed in Chapter \ref{ChapterAggFlexForecast}. When the aggregator receives a flexibility request from the DSO, the aggregator either accepts or rejects it based on the previously calculated available flexibility. If the request is accepted, the scheduling within that portfolio has to be performed, in order to send the control signals to the specific flexible assets that will provide the flexibility at that time period. Consequently, once the control signals are sent to each HEMS unit, another optimization can happen in order to consider an end-user optimization problem based on end-user preferences. The last step of this flexibility chain is the flexibility activation based on the request sent by the DSO.

\vspace*{8mm}

\begin{figure}[htbp]
	\centering
	\includegraphics[width=1\columnwidth ]{ChapterFlexServices/Figures/AGG-PROS-DSO_3.pdf}
	   %\vspace*{-8cm}
		\caption{Interaction between aggregators and prosumers for flexibility provision.}
	\label{fig:agg-pros-dso}  
\end{figure}

\newpage
\section{Chapter remarks}

In this chapter, the current framework about the flexibility provision in power systems has been outlined, considering all the agents that can provide flexibility, all the clients that can use this flexibility, all the schemes to exchange this service, and all the different approaches and formulations to define this service. The main focus is set on DSOs, demand-side, and aggregators. The framework presented in this chapter allows the definition of the flexibility signal that will be forecast and provided to the DSO for active grid management under operation time-horizon. This is done by the aggregator figure's participation, responsible for managing and controlling a portfolio of flexible assets with different nature. It is essential to define the flexibility signal based on the specific agents that will participate in this exchange, as well as the related data, monitoring systems, and controllability of the flexible assets, to evaluate the flexibility activated in a specific portfolio. The following chapters will use the flexibility formulation defined here for forecasting the available flexibility within an aggregator's portfolio in Chapter \ref{ChapterAggFlexForecast} and the calculation of the flexibility request in Chapter \ref{ChapterOPFDSO}. 



