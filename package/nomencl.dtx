% \iffalse
% File nomencl.dtx
% Copyright 1996 Boris Veytsman
% Copyright 1999-2001 Bernd Schandl
% www   http://sarovar.org/projects/nomencl
%
% Documentation and documented source code for the nomencl package.
%
% See the README file for instructions.
%
% This file can be redistributed and/or modified under the terms
% of the LaTeX Project Public License distributed from CTAN
% archives in the directory macros/latex/base/lppl.txt; either
% version 1.2 of the license, or (at your option) any later version.
%
% \fi
% \CheckSum{0}
%% \CharacterTable
%%  {Upper-case    \A\B\C\D\E\F\G\H\I\J\K\L\M\N\O\P\Q\R\S\T\U\V\W\X\Y\Z
%%   Lower-case    \a\b\c\d\e\f\g\h\i\j\k\l\m\n\o\p\q\r\s\t\u\v\w\x\y\z
%%   Digits        \0\1\2\3\4\5\6\7\8\9
%%   Exclamation   \!     Double quote  \"     Hash (number) \#
%%   Dollar        \$     Percent       \%     Ampersand     \&
%%   Acute accent  \'     Left paren    \(     Right paren   \)
%%   Asterisk      \*     Plus          \+     Comma         \,
%%   Minus         \-     Point         \.     Solidus       \/
%%   Colon         \:     Semicolon     \;     Less than     \<
%%   Equals        \=     Greater than  \>     Question mark \?
%%   Commercial at \@     Left bracket  \[     Backslash     \\
%%   Right bracket \]     Circumflex    \^     Underscore    \_
%%   Grave accent  \`     Left brace    \{     Vertical bar  \|
%%   Right brace   \}     Tilde         \~}
%%
% \iffalse
%<*dtx>
          \ProvidesFile{nomencl.dtx}%
%</dtx>
%<package>\ProvidesPackage{nomencl}%
%<driver>\ProvidesFile{nomencl.drv}%
%<*package|driver>
% \fi
%         \ProvidesFile{nomencl.dtx}%
          [2005/09/22 v4.2 Nomenclature package (LN)]
%
% \iffalse
%</package|driver>
%<*driver>
\documentclass[11pt]{ltxdoc}%
\RequirePackage{url}%
\RequirePackage[dvips]{hyperref}%
\RequirePackage[nocfg]{nomencl}%

\hypersetup{%
    hyperindex=true,
    colorlinks=true,
    linkcolor=blue,
    anchorcolor=blue,
    citecolor=blue,
    filecolor=blue,
    menucolor=blue,
    pagecolor=blue,
    urlcolor=blue,
    bookmarksnumbered=true,
    pdftitle={Documentation for the nomencl Package},
    pdfauthor={Boris Veytsman, Bernd Schandl, Lee Netherton, CV
    Radhakrishnan},
    pdfsubject={Provides nomenclature automation in LaTeX},
    pdfkeywords={nomenclature latex generation symbol list}
}%

%% Comment out the next line if you want the documentation for the source code.
%%\OnlyDescription

%% Uncomment the next two lines for a change history.
%% See also the instructions above (in nomencl.drv).
%% \AtBeginDocument{\RecordChanges}
%% \AtEndDocument{\addcontentsline{toc}{section}{Change Log}\PrintChanges}

%% Uncomment the next two lines for a command index.
%% See also the instructions above (in nomencl.drv).
\AtBeginDocument{\CodelineIndex}
\AtEndDocument{\setcounter{IndexColumns}{2}\addcontentsline{toc}{section}{Index}\PrintIndex}

\begin{document}
  \DocInput{nomencl.dtx}
\end{document}
%</driver>
% \fi
%
%
% \changes{v2.2 (1996/11/25)}{1996/11/25}{Last version released by
%   Boris Veytsman}
% \changes{v2.5 (1999/03/13)}{1999/03/02}{Complete rewrite of the
%   package and the documentation}
% \changes{v2.6 (1999/04/02)}{1999/04/01}{Use \cs{GetFileInfo}}
% \changes{v2.7a (1999/07/07)}{1999/07/07}{Merged \texttt{licence.txt}
%   into \texttt{README}}
% \changes{v2.8 (1999/09/09)}{1999/09/09}{Email changed}
% \changes{v3.0 (2000/03/05)}{2000/03/05}{WWW address changed}
% \changes{v3.1 (2000/09/15)}{2000/09/15}{Do not read cfg file in
%   documentation}
% \changes{v3.1 (2000/09/15)}{2000/09/15}{WWW address changed (again)}
% \changes{v3.1b (2001/09/30)}{2001/09/30}{WWW address changed (again)}
% \changes{v3.1c (2001/10/02)}{2001/10/02}{Minor documentation changes}
% \changes{v4.0 (2005/03/31)}{2005/03/31}{Improved compatibility with other
% Glossary/MakeIndex packages. Added option to insert Nomenclature into toc.
% Amended documentation accordingly.}
% \changes{v4.1 (2005/04/27)}{2005/04/27}{Improvements to the documentation,
% including hyperref support}
%
% \GetFileInfo{nomencl.dtx}
%
%
% \MakeShortVerb{\|}
% \setlength{\hfuzz}{8pt}
% \newcommand{\MakeIndex}{\textsl{MakeIndex}}
% \newcommand{\xindy}{\mbox{\normalfont\textbf{\textsf{x\kern-0.7pt%
%         \shortstack{{\scriptsize$\circ$}\\[-2pt]\i}\kern-1pt ndy}}}}
% \let\package\textsf
% \let\tab\indent
% \newenvironment{example}
%   {\begin{figure}[htbp]\rule[2pt]{\linewidth}{1pt}}
%   {\rule[2pt]{\linewidth}{1pt}\end{figure}}
%
% \title{\package{nomencl}\thanks{Package version \fileversion\ of \filedate.}
%   \\ A Package to Create a Nomenclature}
% \author{Boris Veytsman\thanks{Up to version v2.2 (1996/11/25)}\\
%       Bernd Schandl\thanks{Up to version v3.1c (2001/10/02)}\\
%       Lee Netherton \& CV Radhakrishnan\thanks{Up to version v4.0 (2005/03/31)}\\\\
%    {\small \href{http://sarovar.org/projects/nomencl}{http://sarovar.org/projects/nomencl}}}
% \date{Printed on \today}
% \maketitle
%
%
% \section{Introduction}
% How often did you try to understand a theorem in a book, but just
% couldn't figure out what all those strange symbols were all about? The
% \package{nomencl} package should help authors format a nomenclature.
% It uses the powerful capabilities of the \MakeIndex\ program to generate
% such a list automatically using information provided by the author
% throughout the text.
%
% \subsection{Important Notes for Users of Previous Versions}
% The latest update to the \package{nomencl} package has included some major changes
% to some of the more important commands. In particular, the |\makeglossary| and
% |\printglossary| commands have now been renamed to |\makenomenclature| and
% |\printnomenclature| respectively. The reason for this change is to increase the
% package's compatibility with other MakeIndex using packages. With this increased
% compatibility, users will be able to have nomenclatures, glossaries and indexes all
% in one document.\\
%
% There is a compatibility option that will allow you to still use
% your |\makeglossary| and |\printglossary| commands (see section \ref{sec:options}),
% but it is advised that you change your |\|\ldots|glossary| commands to the new
% |\|\ldots|nomenclature| commands in your \LaTeX\ files. For more information
% on the compatibility mode see section \ref{sec:compatibility}.\\
%
% \section{Usage}
%
% \subsection{The Basics}
%
% The creation of the nomenclature list is very similar to the creation of
% an index~\cite[App.~A]{lamp:late:1994}. You need to:
% \begin{itemize}
% \item Put |\usepackage[|\meta{options}|]{nomencl}| in the preamble of
%   your document.
% \item \DescribeMacro{\makenomenclature}
%   Put |\makenomenclature| in the preamble of your document.
% \item \DescribeMacro{\nomenclature}
%   Issue the |\nomenclature| command (see Section~\ref{sec:Main-Command})
%   for each symbol you want to have included in the nomenclature list.
%   The best place for this command is immediately after you introduce
%   the symbol for the first time.
% \item \DescribeMacro{\printnomenclature}
%   Put |\printnomenclature| at the place you want to have your
%   nomenclature list.
% \end{itemize}
%
% Now put your file through \LaTeX. The command |\makenomenclature| will
% instruct \LaTeX\ to open the nomenclature file \meta{filename}|.nlo|
% corresponding to your \LaTeX\ file \meta{filename}|.tex| and to
% write the information from your |\nomenclature| commands to this file.
%
% \changes{v2.7 (1999/05/14)}{1999/05/11}{Mention need to change quote
%   character for German users}
% The next step is to invoke \MakeIndex. You should instruct
% \MakeIndex\ to use \meta{filename}|.nlo| as your input file,
% use |nomencl.ist| as your style
% \DeleteShortVerb{\|}\MakeShortVerb{\?}%
% file\footnote{%
%  German users who want to use the shortcut notation ?"a? instead
%  of \cs{"a} have to redefine the quote character in ?nomencl.ist?
%  to something other than ?"? (and ?|?, ?@?, ?!?), maybe ?+? or ?\&?; see the
%  comment in the source code section and in the file ?nomencl.ist?.
%  Furthermore, they should consider using the ?-g? switch of \MakeIndex.},
% \DeleteShortVerb{\?}\MakeShortVerb{\|}%
% and write output to the file \meta{filename}|.nls|.
% How to do this depends on your implementation of \MakeIndex. For
% most UNIX implementations you should write something like\\
% \tab|makeindex| \meta{filename}|.nlo -s nomencl.ist -o|
%   \meta{filename}|.nls|
%
% Now you have the file \meta{filename}|.nls| that contains your
% nomenclature list properly ordered. The last step is to invoke
% \LaTeX\ on your master file \meta{filename}|.tex| once more. It will
% input your |.nls| file and process it accordingly to the current
% options. That's all!
%
%
% \subsection{The Main Command}
% \label{sec:Main-Command}
%
% \DescribeMacro{\nomenclature}
% The main command of the \package{nomencl} package has the following
% syntax:\\
% \tab|\nomenclature[|\meta{prefix}|]{|\meta{symbol}|}{|\meta{description}|}|\\
% where \meta{prefix} is used for fine tuning the sort order,
% \meta{symbol} is the symbol you want to describe and \meta{description}
% is the actual description. The sortkey will be \meta{prefix}\meta{symbol},
% where \meta{prefix} is either the one from the optional argument or, if no
% optional argument was given, the default \meta{prefix} which may be empty.
% See Section~\ref{sec:sort} to make sense of this.
%
% Put this command immediately after the equation or text that
% introduces \meta{symbol}. Usually it is a good idea to avoid a space
% or an unquoted newline just in front of the |\nomenclature| command.
% Put a |%| at the end of the preceding line if necessary.
% Don't forget to enclose math in \meta{symbol} in |$| signs.
%
% Let's have a look at a simple example. If your input file looks like the
% one in Figure~\ref{fig:simple} then your nomenclature\footnote{Note that
% all the examples are somewhat faked in this document, but they give a good
% impression of the ``real'' result.} should look like
% Figure~\ref{fig:simple.out}.
%
% \begin{example}
%\begin{verbatim}
% \documentclass{article}
% \usepackage{nomencl}
% \makenomenclature
% \begin{document}
% \section*{Main equations}
% \begin{equation}
%   a=\frac{N}{A}
% \end{equation}%
% \nomenclature{$a$}{The number of angels per unit area}%
% \nomenclature{$N$}{The number of angels per needle point}%
% \nomenclature{$A$}{The area of the needle point}%
% The equation $\sigma = m a$%
% \nomenclature{$\sigma$}{The total mass of angels per unit area}%
% \nomenclature{$m$}{The mass of one angel}
% follows easily.
% \printnomenclature
% \end{document}
%\end{verbatim}
% \caption{Input of a simple example}
% \label{fig:simple}
% \end{example}
% \begin{example}
% \begin{thenomenclature}
% \nomgroup{A}
%  \item[$\sigma$]\begingroup The total mass of angels per unit area\nomeqref {1}
%                \nompageref{1}
%  \item[$A$]\begingroup The area of the needle point\nomeqref {1}
%                \nompageref{1}
%  \item[$a$]\begingroup The number of angels per unit area\nomeqref {1}
%                \nompageref{1}
%  \item[$m$]\begingroup The mass of one angel\nomeqref {1}
%                \nompageref{1}
%  \item[$N$]\begingroup The number of angels per needle point\nomeqref {1}
%                \nompageref{1}
% \end{thenomenclature}
% \caption{Output of a simple example}
% \label{fig:simple.out}
% \end{example}
%
% Note the necessary quoting of newlines. When the |\nomenclature|
% macros appear directly after the |equation| environment, quote \emph{all}
% newlines; when they appear in the middle of a sentence, quote \emph{all but
% the last} newlines.\footnote{I'm not sure how to resolve this more
%   elegantly, but suggestions are welcome.}
%
% \changes{v2.6b (1999/04/10)}{1999/04/08}{Documentation change concerning line
%   breaks between arguments}
% Due to the way \cs{nomenclature} scans its arguments you don't need to
% \cs{protect} any macros, but you also must not have any character in front
% of the first or between the
% first and the second argument, especially no line break (even with a |%|). So
%\begin{verbatim}
%   \nomenclature{$x$}%
%     {Description}
%\end{verbatim}
% does \emph{not} work. You can have have line breaks in the argument, but
% also no |%|.
%
%
% \subsection{Package Options}
% \label{sec:options}
%
% The \package{nomencl} package has the following options:
% \begin{description}
% \item[refeq] The phrase ``, see equation (\meta{eq})'' is appended to
%   every entry in the nomenclature where \meta{eq} is the number of the
%   last equation in front of the corresponding command |\nomenclature|.
% \item[norefeq] No equation reference is printed. (default)
% \item[refpage] The phrase ``, page \meta{page}'' is appended to
%   every entry in the nomenclature where \meta{page} is the number of the
%   page on which the corresponding command |\nomenclature| appeared.
% \item[norefpage] No page reference is printed. (default)
% \item[prefix] Every sort key is preceded by the letter ``a'' (changeable);
%   see Section~\ref{sec:sort} to learn why this might make sense. (default)
% \item[noprefix] No prefix is used.
% \changes{v3.0 (2000/03/05)}{2000/03/05}{New options
%   \texttt{cfg}/\texttt{nocfg}}
% \item[cfg] A configuration file |nomencl.cfg| is loaded, if it exists.
%   (default)
% \item[nocfg] The configuration file is not loaded.
% \item[intoc] Inserts the nomenclature in the Table of Contents.
% \item[notintoc] No entry for the nomenclature in the Table of Contents. (default)
% \item[compatible] Run in compatibility mode. Older tex files may need this option
%   selected to be able to compile. In the latest version of \package{nomencl} the commands |\makeglossary|
%   and |\printglossary| were replaced with |\makenomenclature| and |\printnomenclature|.
%   Selecting this option will redefine the old commands, but will loose the compatibility
%   with other glossary packages.
% \item[noncompatible] Do not run in compatibility mode. (default)
% \item[\hspace{-\labelsep}]
%   \textbf{croatian, danish, english, french, german, italian, polish,
%     portuguese, russian, spanish, ukrainian}
%   The reference texts and the nomenclature title will appear in the
%   corresponding language. Note that in order to use Russian or Ukrainian,
%   you have to have Cyrillic fonts installed and you might need a replacement
%   for \MakeIndex, e.\,g.~\xindy. Please help me out with other
%   languages. (default: english)
% \end{description}
%
%
% \subsection{Referencing}
%
% \DescribeMacro{\nomrefeq}\DescribeMacro{\nomrefpage}\DescribeMacro{\nomrefeqpage}
% \DescribeMacro{\nomnorefeq}\DescribeMacro{\nomnorefpage}
% \DescribeMacro{\nomnorefeqpage}
% As explained in Section~\ref{sec:options}, you can turn referencing to
% equations and pages on/off globally using the package options. But sometimes
% you might want to change the referencing behavior for single entries. The
% following six macros can be used inside a |\nomenclature| macro:
% |\nomrefeq|, |\nomnorefeq|, |\nomrefpage|, |\nomnorefpage|, |\nomrefeqpage|,
% |\nomnorefeqpage|.
% The first four work similarly to the package options, only local to the
% entry; the last two are shortcuts, so saying |\nomrefeqpage| is equivalent
% to |\nomrefeq\nomrefpage|.
%
% If we changed the relevant parts of the last example as shown in
% Figure~\ref{fig:reference} then the nomenclature should look like
% Figure~\ref{fig:reference.out}.
%
% \begin{example}
%\begin{verbatim}
% \begin{equation}
%   a=\frac{N}{A}
% \end{equation}%
% \nomenclature{$a$}{The number of angels per unit area\nomrefeqpage}%
% \nomenclature{$N$}{The number of angels per needle point\nomrefeq}%
% \nomenclature{$A$}{The area of the needle point\nomrefeq\nomrefpage}%
% The equation $\sigma = m a$%
% \nomenclature{$\sigma$}{The total mass of angels per unit area}%
% \nomenclature{$m$}{The mass of one angel\nomrefpage}
% follows easily.
% \printnomenclature
% \end{document}
%\end{verbatim}
% \caption{Input with references}
% \label{fig:reference}
% \end{example}
% \begin{example}
% \begin{thenomenclature}
% \nomgroup{A}
%  \item [$\sigma$]\begingroup The total mass of angels per unit area\nomeqref {1}
%                \nompageref{1}
%  \item [$A$]\begingroup The area of the needle point\nomrefeq\nomrefpage\nomeqref {1}
%                \nompageref{1}
%  \item [$a$]\begingroup The number of angels per unit area\nomrefeqpage\nomeqref {1}
%                \nompageref{1}
%  \item [$m$]\begingroup The mass of one angel\nomrefpage\nomeqref {1}
%                \nompageref{1}
%  \item [$N$]\begingroup The number of angels per needle point\nomrefeq\nomeqref {1}
%                \nompageref{1}
% \end{thenomenclature}
% \caption{Output with references}
% \label{fig:reference.out}
% \end{example}
%
% While these macros do not have to be at the end of the entries, it's
% probably the most sensible place to put them. Note that such local request
% always supersede the package options.
%
%
% \section{Sort Order of the Entries}
% \label{sec:sort}
%
% The Greek letter $\sigma$ turned out to be first in the nomenclature
% list in the examples above because the backslash in |\sigma|
% precedes any alphabetical character. Sometimes this is not what you
% want.  Then you can use \meta{prefix} to fine tune the sort order.
%
% Before we describe the usage of \meta{prefix}, we have to explain how
% \MakeIndex\ sorts entries, see~\cite{chen:auto:1987}. \MakeIndex\
% distinguishes three kinds of sort keys:
% \begin{description}
% \item[Strings] Everything that starts with a alphabetic letter (A\dots Z,
%   a\dots z).
% \item[Numbers] Everything that starts and only contains digits (0\dots 9).
% \item[Symbols] Everything else.
% \end{description}
% Each group is sorted separately (and differently), then the groups are
% sorted in the order symbols, numbers, strings\footnote{With the |-g| switch
% of \MakeIndex, they are sorted in the order symbols, strings, numbers.}. For
% the groups the following algorithm\footnote{This is only vaguely described
% in~\cite{chen:auto:1987}, so I had to figure out special cases by
% myself. Please correct me if I am wrong} is used:
% \begin{description}
% \item[Strings] If two letters are compared, the usual ordering is used
%   (|a|\textless|C|\textless|q|), but if two words are the
%   same except for the capitalization, then an upper case letter
%   precedes the lower case letter (|Tea|\textless|tea|).
%   If a letter is compared with a non-letter (digit,
%   symbol), ASCII code is used (|1|\textless |A|\textless |~|).\footnote{An
%   exception seems to be that the non-letters between upper and lower case
%   letters (code 91--96) are put just before the capital letters (between
%   code 64 and 65) while the non-letters after the lower case letters (code
%   123--127) are left there. Can someone please enlighten me why?} If two
%   non-letters are compared (which can not happen at the first position of a
%   string), ASCII code is used (|+|\textless |1|\textless |:|\textless
%   |\|). Additionally there is the issue of word ordering (treat spaces as
%   letters with ASCII code smaller than every printable symbol) and letter
%   ordering (ignore spaces). \MakeIndex\ uses word ordering by default, but
%   you can change it with some command line option (|-l| on my UNIX).
% \item[Numbers] The natural ordering is used
%   (|8|\textless |34|\textless |111|).
% \item[Symbols] ASCII code is used
%   (|+|\textless|1|\textless|:|\textless|A|\textless|\|\textless|a|).
% \end{description}
%
% Why did you have to read all this?\footnote{I hope you did read it
%   \texttt{;-)}} Let's consider the following eight nomenclature entries
% (without the optional argument): |$~Ab$|, |$~aa$|, |$\Ab$|, |$\aa$|, |$Ab$|,
% |$aa$|, |Ab|, |aa|. Try to understand the following example with the help
% of the explanation above and an ASCII table.
%
% If you use \package{nomencl} with its default settings (i.\,e.~``a'' is
% added to every sort key, so every sort key is considered as a string), you
% will get the sort order |$\aa$|, |$\Ab$|, |$aa$|, |$Ab$|, |$~aa$|, |$~Ab$|,
% |aa|, |Ab|. Note that |aa| is in front of |Ab| in all four pairs; note
% also the order |$\Ab$|, |$Ab$|, |$~Ab$| which does not agree with the
% ASCII code.
%
% If you specify the option \package{noprefix}, then you will get |$Ab$|,
% |$\Ab$|, |$\aa$|, |$aa$|, |$~Ab$|, |$~aa$|, |aa|, |Ab|. The first six
% entries are considered as symbols and sorted according to the ASCII code
% (this time correctly). Note that |$\Ab$| is in front of |$\aa$| because
% |A| has the smaller ASCII code. The two strings follow at the end.
%
% Decide for yourself what you prefer. Personally, I like to specify the
% \package{noprefix} option and use the optional argument to get exactly the
% sort order I want. See Section~\ref{sec:tips} for some special effects.
%
%
% \section{Customization}
% \label{sec:custom}
%
% Besides the things you can customize by using the package options, there are
% a few more commands that you might want to redefine. If you make the same
% changes in every file, it's probably easier to put all those in a file
% |nomencl.cfg| which is automatically read by the \package{nomencl} package
% whenever it exists in the search path (unless you specified the |nocfg|
% option).
%
%
% \subsection{Formatting the Nomenclature}
% \label{sec:format}
%
% \DescribeMacro{\printnomenclature}
% \DescribeMacro{\nomlabelwidth}
% Probably the most common change to the nomenclature is a different amount
% of space for the symbols. By default, the nomenclature is formatted as a
% list with the label width equal to |\nomlabelwidth| which is initialized
% to 1\,cm. You can change this dimension in the |cfg| file or you can use
% the optional argument of |\printnomenclature|. If you want to have a little
% more space for the labels (and you don't live in a metric world) you can
% use\\
% \tab|\printnomenclature[0.5in]|\\
% instead of the simple\\
% \tab|\printnomenclature|
%
% \DescribeEnv{thenomenclature}
% If you don't like the format of the nomenclature at all, you will have
% to redefine the |thenomenclature| environment. Maybe a look at the
% documented code of \package{nomencl} will help.
%
% \DescribeMacro{\nomname}
% In case you don't like the name of the nomenclature, just redefine
% the |\nomname| macro, e.\,g.\\
% \tab|\renewcommand{\nomname}{List of Symbols}|\\
% If you are using e.\,g.~the documentclass |book| with page style
% headings you should also take care of correct headings:
%\begin{verbatim}
%   \cleardoublepage% or \clearpage
%   \markboth{\nomname}{\nomname}% maybe with \MakeUppercase
%   \printnomenclature
%\end{verbatim}
% I thought about putting this in the definition of |\printnomenclature|
% but decided that it is much easier for the user to add it if he wants
% than to remove it if he doesn't want it. In case you always need this
% just define a macro in |nomencl.cfg| that executes these three lines
% all at once and can be used instead of |\printnomenclature|.
%
% \changes{v3.1b (2001/09/30)}{2001/03/12}{Explain how to get toc entry}
% \changes{v4.0 (2005/03/31)}{2005/03/31}{TOC entries now added with package option}
% Putting an entry for the nomenclature in the table of contents can
% be done by adding an \textbf{intoc} to the package options.
%
% \DescribeMacro{\nomgroup}
% Usually, \MakeIndex\ inserts the macro |\indexspace| between every
% character group, i.\,e.~between symbols and numbers, numbers and
% letters and between every two letter groups. The \package{nomencl}
% package inserts the macro |\nomgroup{|\meta{arg}|}| \emph{instead},
% where \meta{arg} is either the string ``Symbols'' or the string
% ``Numbers'' or the capital letter of the group that is about to
% start. You can redefine |\nomgroup| to insert some white space\\
% \tab|\renewcommand{\nomgroup}[1]{\medskip}|\\
% or to print a fancy divider
%\begin{verbatim}
%   \renewcommand{\nomgroup}[1]{%
%     \item[]\hspace*{-\leftmargin}%
%     \rule[2pt]{0.45\linewidth}{1pt}%
%     \hfill #1\hfill
%     \rule[2pt]{0.45\linewidth}{1pt}}
%\end{verbatim}
% Note that |\nomgroup| is executed in a list environment, so you need
% to have an |\item| first and then jump back to the beginning of the
% line with the |\hspace| command.
%
% \DescribeMacro{\nompreamble}\DescribeMacro{\nompostamble}
% Maybe you want to explain something just between the title of the
% nomenclature and the start of the list or at the very end of the
% list. Just redefine the macros |\nompreamble| and |\nompostamble|
% which do nothing by default. Note that they are executed \emph{outside}
% of the list environment.
%
% \DescribeMacro{\nomitemsep}
% The skip between two entries in the nomenclature can be adjusted using
% \cs{nomitemsep}. This should be done in the preamble or the file
% |nomencl.cfg|. Note that if you want no extra skip between entries
% you have to use\\
% \tab|\setlength{\nomitemsep}{-\parsep}|
%
% \DescribeMacro{\nomprefix}
% If you want, you can redefine the default prefix that is used for the
% sortkeys. By default, |\nomprefix| is set to ``a''; redefining it
% supersedes the package options \package{prefix} and \package{noprefix}.
%
%
% \subsection{Formatting the Entries}
%
% \DescribeMacro{\nomlabel}
% By default, the labels are just shifted to the left within their
% allocated box. If you want to change this, redefine |\nomlabel|
% which should get one argument, e.\,g.\\
% \tab|\renewcommand{\nomlabel}[1]{\hfil #1\hfil}|\\
% to center the symbols.
%
% \DescribeMacro{\nomentryend}
% Maybe you would like to have a period at the end of every entry.
% Just say\\
% \tab|\renewcommand{\nomentryend}{.}|\\
% and there it is. Section~\ref{sec:tips:units} explains another nice
% application of this macro.
%
% \DescribeMacro{\eqdeclaration}\DescribeMacro{\pagedeclaration}
% If you don't like the text that is used for the references to  equations
% and pages, you can define |\eqdeclaration| and |\pagedeclaration|. Both
% should accept one argument, namely the equation and page number,
% respectively. An example is\\
% \tab|\renewcommand{\eqdeclaration}[1]{, first used in eq.~(#1)}|\\
% If you are redefining these macros for a particular language, let me
% know and I will add that language to the next release of the
% \package{nomencl} package.
%
%
% \section{Tips and Tricks}
% \label{sec:tips}
%
% \changes{v3.1 (2000/09/15)}{2000/09/15}{Sample cfg files for most examples}
% In this section, I will gather fancy stuff that people did or might
% want to do with the \package{nomencl} package. Please email any ideas
% you have.
%
% For most examples, sample configuration files will be generated if you run
% \LaTeX\ on the file |nomencl.ins|. There will for example be a file
% |sample01.cfg| for the subgroups example in
% Section~\ref{sec:tips:subgroups}. Rename it to |nomencl.cfg|, then it will
% automatically be used by your document. There is no sample file for the
% longtable example in Section~\ref{sec:tips:longtable}. I am just too lazy
% right now, maybe I will add it later\dots
%
%
% \subsection{Subgroups}
% \label{sec:tips:subgroups}
%
% If you have distinct groups among the identifiers in your nomenclature
% (e.\,g.~Greek letters for physical constants, Roman letters for
% variables), you can use the optional argument of |\nomenclature|
% together with the |\nomgroup| macro to get two groups with separate
% headings in the nomenclature.
%
% Use something like the following throughout your text\\
% \tab|\nomenclature[ga ]{$\alpha$}{Constant}|\\
% \tab|\nomenclature[rx ]{$x$}{Variable}|\\
% where ``g'' and ``r'' indicate Greek and Roman letters, respectively.
% Then you include the \package{ifthen} package and redefine |\nomgroup|
% e.\,g.~like this.
%    \begin{macrocode}
%<*sample01>
\RequirePackage{ifthen}
\renewcommand{\nomgroup}[1]{%
  \ifthenelse{\equal{#1}{R}}{\item[\textbf{Variables}]}{%
    \ifthenelse{\equal{#1}{G}}{\item[\textbf{Constants}]}{}}}
%</sample01>
%    \end{macrocode}
% Note that we have to check for capital letters. All your symbols should
% have some kind of prefix; maybe you can also use the default prefix ``a''.
% Note that for symbols and numbers you have to check for the strings
% ``Symbols'' and ``Numbers''.
%
%
% \subsection{Units}
% \label{sec:tips:units}
%
% Besides the obvious possibility of adding units for symbols in the
% description string, you can also use |\nomentryend| to shift the unit
% to the right margin. Something along the lines of a macro
%    \begin{macrocode}
%<*sample02>
\newcommand{\nomunit}[1]{%
  \renewcommand{\nomentryend}{\hspace*{\fill}#1}}
%</sample02>
%    \end{macrocode}
% should do the job.
% You can use this macro like this\\
% \tab|\nomenclature{$l$}{Length\nomunit{m}}|\\
% Note that the nomenclature will not be a tabular with three columns,
% but it is pretty close as long as you only have one-line descriptions.
% Any suggestions for improvements are welcome.
%
%
% \subsection{Using a Long Table instead of a List}
% \label{sec:tips:longtable}
%
% \changes{v2.6 (1999/04/02)}{1999/04/01}{Longtable example added}
% The following idea was sent to me by Brian Elmegaard. I have modified it a
% little bit to make it work with the current version of \package{nomencl}.
% Only the basic idea is given, so you have to do some extra thinking
% (and coding) to get it to work the way you want it.
%
% After loading the \package{longtable} package in the preamble we first
% have to modify the macro that writes the entries to the |glo|
% file (do this is an style file).
%\begin{verbatim}
%    \def\@@@nomenclature[#1]#2#3{%
%     \def\@tempa{#2}\def\@tempb{#3}%
%     \protected@write\@nomenclaturefile{}%
%      {\string\nomenclatureentry{#1\nom@verb\@tempa @{\nom@verb\@tempa}&%
%          \begingroup\nom@verb\@tempb\protect\nomeqref{\theequation}%
%            |nompageref}{\thepage}}%
%     \endgroup
%     \@esphack}
%\end{verbatim}
% Then the nomenclature itself must be changed to start a longtable instead
% of a list. Maybe we could add something for a repeating header on every page.
%\begin{verbatim}
%    \def\thenomenclature{%
%      \@ifundefined{chapter}{\section*}{\chapter*}{\nomname}%
%      \nompreamble
%      \begin{longtable}[l]{@{}ll@{}}}
%    \def\endthenomenclature{%
%      \end{longtable}%
%      \nompostamble}
%\end{verbatim}
% Finally we add the following two lines at the end of |nomencl.ist|\footnote{%
%   Don't forget to rename the file and delete my email address if you want to
%   distribute the file, see the pointer to the LPPL in
%   Section~\ref{sec:legal}.}.
%\begin{verbatim}
%    item_0  ""
%    delim_t " \\\\\n"
%\end{verbatim}
%
% As I said, this is only the basic idea. An advantage might be the repeating
% headers on every page, a disadvantage is that there won't be any line breaks
% in the second column.
%
%
% \subsection{I want it expanded!}
%
% \changes{v3.1 (2000/09/15)}{2000/09/01}{Expansion example added}
% The \package{nomencl} package tries hard to write the arguments of the
% |\nomenclature| macro verbatim to the glossary file. This is usually the
% right thing to do because some macros do not like to be expanded at the
% wrong moment or give weird results if they are. On the other hand, there are
% occasions where it is good to have the meaning (or expansion) of a macro in
% the glossary file instead of its name. There are quite some occasions
% where you will get in trouble with this expansion, for example, if the
% expansion of a macro contains~|@| (|\mathcal| expands to |\@mathcal|)
% because |@| is a special character for \MakeIndex\ and thus \MakeIndex\ will
% either fail or give unexpected results. You can avoid the expansion on a
% case by case basis by using |\protect| in front of the macro that should
% not be expanded.
%
% In order to get macro expansion, the redefinition of the |\@nomenclature|
% macro within the |\makenomenclature| macro has to be changed.
%    \begin{macrocode}
%<*sample04>
\def\makenomenclature{%
  \newwrite\@nomenclaturefile
  \immediate\openout\@nomenclaturefile=\jobname\@outputfileextension
  \def\@nomenclature{%
    \@ifnextchar[%
      {\@@@@nomenclature}{\@@@@nomenclature[\nomprefix]}}%
  \typeout{Writing nomenclature file \jobname\@outputfileextension}%
  \let\makenomenclature\@empty}
%    \end{macrocode}
% The new macro to be called by |\@nomenclature| just writes its arguments to
% the glossary file without further ado, so they will be expanded.
%    \begin{macrocode}
\def\@@@@nomenclature[#1]#2#3{%
 \protected@write\@nomenclaturefile{}%
  {\string\nomenclatureentry{#1#2@[{#2}]%
      \begingroup#3\protect\nomeqref{\theequation}%
        |nompageref}{\thepage}}}%
%</sample04>
%    \end{macrocode}
% As I said above, use these macros with care and look for warnings and errors
% issued by \MakeIndex.
%
%
% \subsection{Glossary in ``Kopka Style''}
%
% \changes{v3.1 (2000/09/15)}{2000/09/16}{Kopka example added}
% I was told that the glossary in the \LaTeX\ book by Kopka looks roughly like
% in Figure~\ref{fig:kopka}. In order to get a glossary like this, there are
% quite some configurations to do.
%
% \begin{example}
% \textbf{Symbol}\dotfill\nopagebreak page number \\
% \hspace*{5mm}Explanation.
% \caption{Glossary entry in ``Kopka Style''}
% \label{fig:kopka}
% \end{example}
%
% First we have to change the macro |\@@@nomenclature| which takes care of
% writing the glossary entry to the glossary file. The only difference to the
% original definition is that we hand over the explanation of a symbol (\#3)
% and the equation number to |\nompageref| instead of writing it directly
% after the symbol (\#2). This is necessary because the explanation should
% appear after (actually below) the page number.
%    \begin{macrocode}
%<*sample05>
\def\@@@nomenclature[#1]#2#3{%
 \def\@tempa{#2}\def\@tempb{#3}%
 \protected@write\@nomenclaturefile{}%
  {\string\nomenclatureentry{#1\nom@verb\@tempa @[{\nom@verb\@tempa}]%
    |nompageref{\begingroup\nom@verb\@tempb\protect\nomeqref{\theequation}}}%
    {\thepage}}%
 \endgroup
 \@esphack}
%    \end{macrocode}
% Now we change the definition of |\nompageref| so that it accepts two
% arguments, the explanation (\#1) and the page number (\#2). The page number
% is only printed if required, otherwise  |\null| is used to avoid an error
% because of the following |\linebreak|. Note that it is \emph{not} possible
% to turn off the page number locally, because the explanation appears after
% the page number. Does anyone have an idea how to fix this?
%    \begin{macrocode}
\def\nompageref#1#2{%
  \if@printpageref\pagedeclaration{#2}\else\null\fi
  \linebreak#1\nomentryend\endgroup}
%    \end{macrocode}
% And a few little things. We want dots and a space before the page number
% appears at the right margin; the explanation should end with a period; and
% the symbol should be printed in bold face (this only works for regular text,
% not for formulas).
%    \begin{macrocode}
\def\pagedeclaration#1{\dotfill\nobreakspace#1}
\def\nomentryend{.}
\def\nomlabel#1{\textbf{#1}\hfil}
%</sample05>
%    \end{macrocode}
%
%
% \section{Compatibility Mode}
% \label{sec:compatibility}
%
% With previous versions of the \package{nomencl}, the commands |\makeglossary|
% and |\pringlossary| were used to generate and display the nomenclature.
% These commands have now been depreciated, and replaced with the |\makenomenclature|
% and |\printnomenclature| commands. The new commands do exactly the same
% as the old commands, but because of the name changes, the package is
% now compatible with other packages which use the |\makeglossary| commands.
% The previous versions of \package{nomencl} also used the file extensions
% |.glo| and |.gls| for the generated output and input files. These extensions
% have now been changed to |.nlo| and |.nls| respectively---again, for increased
% compatibility.
%
% For all of the legacy \LaTeX\ files out there which use the old commands
% there is a compatibility option available so that the old commands will still
% work without having to change any of the existing code. To enable the
% compatibility mode simply supply the \textbf{compatible} option
% when using the package. For example:
% \begin{verbatim}
%   \usepackage[compatible]{nomencl}
% \end{verbatim}
% Under compatibility mode, the package will generate and use files
% with the old-style file extensions (i.e. |.glo| and |.gls|).
%
% It is worth noting that even though the compatibility mode is available,
% it is highly recommended to update your \LaTeX\ files to use the new
% nomenclature commands.
%
%
% \section{Acknowledgements}
%
% First and foremost I want to thank Boris Veytsman, who had the idea
% for the package, maintained it until v2.2 and provided some
% helpful advice for the new version. I also want to thank Stefan
% B\"ohm and Karl Heinz Marbaise who helped testing this package.
%
% For helping out with translations I thank Branka Lon\v{c}arevi\'{c}
% (Croatian), Brian Elmegaard (Danish), Denis B.~Roegel (French),
% Sani Egisto (Italian), Artur Gorka (Polish), Pedro Areal (Portuguese),
% Alejandro Lopez-Valencia (Spanish) and Boris Veytsman
% (Russian and Ukrainian).
%
%
% \section{Releases and Legal Issues}
% \label{sec:legal}
%
% This package can be redistributed and/or modified under the terms
% of the \LaTeX\ Project Public License distributed from CTAN
% archives in the directory \url{macros/latex/base/lppl.txt}, see
% e.\,g.~\cite{ctan}; either
% version 1.2 of the license, or (at your option) any later version.
%
% The most recent release of the \package{nomencl} package can always
% be found at \url{http://sarovar.org/projects/nomencl}.
% Usually, the same version is also available at
% \url{CTAN/macros/latex/contrib/supported/nomencl/}.
%
% \StopEventually{%
% \begin{thebibliography}{5}
% \bibitem[1]{braa:ltid:1996}
% Braams, Johannes; Carlisle, David; Jeffrey, Alan; Lamport, Leslie;
% Mittelbach, Frank; Rowley, Chris; Sch\"opf, Rainer (1996).
% \newblock\texttt{ltidxglo.dtx} -- 1996/01/20 v1.1e LaTeX Kernel
% (Index and Glossary).
% \newblock \href{http://www.ctan.org/tex-archive/macros/latex/base/ltidxglo.dtx}{CTAN/macros/latex/base/ltidxglo.dtx}.
% \bibitem[2]{chen:auto:1987}
% Chen, Pehong; Harrison, Michael~A. (1987).
% \newblock Automating Index Preparation.
% \newblock Report UCB/CSD 87/347, Computer Science Division, University of
%   California, Berkeley, CA.
% \bibitem[3]{ctan}
% Comprehensive {\TeX} Archive Network CTAN.
% \newblock \url{ftp://ctan.tug.org/tex-archive/}.
% \bibitem[4]{jone:anew:1995}
% Jones, David M. (1995).
% \newblock A new implementation of \LaTeX's indexing commands,
%   Version v4.1beta of 1995/09/28.
% \newblock \href{http://www.ctan.org/tex-archive/macros/latex/contrib/camel/index.dtx}{CTAN/macros/latex/contrib/supported/camel/index.dtx}.
% \bibitem[5]{knut:thet:1984}
% Knuth, Donald E. (1984).
% \newblock \emph{The \TeX book}.
% \newblock Addison-Wesley Publishing Company, Reading, MA.
% \bibitem[6]{lamp:late:1994}
% Lamport, Leslie (1994).
% \newblock \emph{{\LaTeX}: A Document Preparation System}.
% \newblock Addison-Wesley Publishing Company, Reading, MA.
% \bibitem[7]{veyt:pack:1996}
% Veytsman, Boris (1996).
% \newblock Package nomencl, Version 4.0.
% \newblock \url{http://sarovar.org/projects/nomencl} (2000/09/15).
% \end{thebibliography}}
%
%
% \section{Implementation}
%
% \subsection{The \LaTeX\ Package File}
%
% At the beginning of this file, the |\ProvidesPackage| macro was executed. So
% we only need to to state that we need \LaTeXe.
%    \begin{macrocode}
%<*package>
\NeedsTeXFormat{LaTeX2e}
%    \end{macrocode}
% \begin{macro}{\if@printeqref}
% \begin{macro}{\if@printpageref}
% We need two switches to decide whether references to equations and pages
% should be printed.
%    \begin{macrocode}
\newif\if@printeqref
\newif\if@printpageref
%    \end{macrocode}
% \end{macro} \end{macro}
% \begin{macro}{\if@intoc}
% Another switch to decide whether to add an entry to the TOC.
%    \begin{macrocode}
\newif\if@intoc
%    \end{macrocode}
% \end{macro}
% \begin{macro}{\if@compatibilitymode}
% Another switch to decide whether to run in compatibility mode.
%    \begin{macrocode}
\newif\if@compatibilitymode
%    \end{macrocode}
% \end{macro}
% And the options to set these switches globally.
%    \begin{macrocode}
\DeclareOption{refeq}{\@printeqreftrue}
\DeclareOption{norefeq}{\@printeqreffalse}
\DeclareOption{refpage}{\@printpagereftrue}
\DeclareOption{norefpage}{\@printpagereffalse}
\DeclareOption{intoc}{\@intoctrue}
\DeclareOption{notintoc}{\@intocfalse}
\DeclareOption{compatible}{\@compatibilitymodetrue}
\DeclareOption{noncompatible}{\@compatibilitymodefalse}
%    \end{macrocode}
% \begin{macro}{\nomprefix}
% It might make sense to add the prefix ``a'' to every sortkey, see
% Section~\ref{sec:sort}.
%    \begin{macrocode}
\DeclareOption{prefix}{\def\nomprefix{a}}
\DeclareOption{noprefix}{\def\nomprefix{}}
%    \end{macrocode}
% \end{macro}
% \begin{macro}{\if@loadcfg}
% Another switch and the corresponding options to decide whether we
% should look for a configuration file.
%    \begin{macrocode}
\newif\if@loadcfg
\DeclareOption{cfg}{\@loadcfgtrue}
\DeclareOption{nocfg}{\@loadcfgfalse}
%    \end{macrocode}
% \end{macro}
% \changes{v2.5a (1999/03/22)}{1999/03/17}{Added Danish}
% \changes{v2.6 (1999/04/02)}{1999/03/23}{Added French}
% \changes{v2.6 (1999/04/02)}{1999/04/01}{Use \cs{nobreakspace} instead of
%   \texttt{\textasciitilde} in package options}
% \changes{v2.6a (1999/04/06)}{1999/04/02}{Added Russian, Spanish, Ukrainian}
% \changes{v2.6b (1999/04/10)}{1999/04/06}{Added Polish}
% \changes{v2.7a (1999/07/07)}{1999/07/07}{Added Italian}
% \changes{v4.0 (2005/04/07)}{2005/04/07}{Updated Italian option (thanks to Lapo Mori)}
% \changes{v3.1 (2000/09/15)}{2000/08/30}{Added Croatian}
% \changes{v3.1a (2000/12/03)}{2000/12/03}{Added Portuguese}
% \begin{macro}{\eqdeclaration}
% \begin{macro}{\pagedeclaration}
% \begin{macro}{\nomname}
% If you can help out with translations for some other languages, let me know.
%    \begin{macrocode}
\DeclareOption{croatian}{%
  \def\eqdeclaration#1{, vidi jednad\v{z}bu\nobreakspace(#1)}%
  \def\pagedeclaration#1{, stranica\nobreakspace#1}%
  \def\nomname{Popis simbola}}
\DeclareOption{danish}{%
  \def\eqdeclaration#1{, se ligning\nobreakspace(#1)}%
  \def\pagedeclaration#1{, side\nobreakspace#1}%
  \def\nomname{Symbolliste}}
\DeclareOption{english}{%
  \def\eqdeclaration#1{, see equation\nobreakspace(#1)}%
  \def\pagedeclaration#1{, page\nobreakspace#1}%
  \def\nomname{Nomenclature}}
\DeclareOption{french}{%
  \def\eqdeclaration#1{, voir \'equation\nobreakspace(#1)}%
  \def\pagedeclaration#1{, page\nobreakspace#1}%
  \def\nomname{Liste des symboles}}
\DeclareOption{german}{%
  \def\eqdeclaration#1{, siehe Gleichung\nobreakspace(#1)}%
  \def\pagedeclaration#1{, Seite\nobreakspace#1}%
  \def\nomname{Symbolverzeichnis}}
\DeclareOption{italian}{%
\def\eqdeclaration#1{, vedi equazione\nobreakspace(#1)}%
\def\pagedeclaration#1{, pagina\nobreakspace#1}%
\def\nomname{Elenco dei simboli}}
\DeclareOption{polish}{%
  \def\eqdeclaration#1{, porownaj rownanie\nobreakspace(#1)}%
  \def\pagedeclaration#1{, strona\nobreakspace#1}%
  \def\nomname{Lista symboli}}
\DeclareOption{portuguese}{%
  \def\eqdeclaration#1{, veja equa\c{c}\~ao\nobreakspace(#1)}%
  \def\pagedeclaration#1{, p\'agina\nobreakspace#1}%
  \def\nomname{Nomenclatura}}
\DeclareOption{russian}{%
  \def\eqdeclaration#1{, \cyrs\cyrm.\nobreakspace(#1)}%
  \def\pagedeclaration#1{, \cyrs\cyrt\cyrr.\nobreakspace#1}%
  \def\nomname{\CYRS\cyrp\cyri\cyrs\cyro\cyrk%
    \ \cyro\cyrb\cyro\cyrz\cyrn\cyra\cyrch\cyre\cyrn\cyri%
    \cyrishrt}}
\DeclareOption{spanish}{%
  \def\eqdeclaration#1{, v\'ease la ecuaci\'on\nobreakspace(#1)}%
  \def\pagedeclaration#1{, p\'agina\nobreakspace#1}%
  \def\nomname{Nomenclatura}}
\DeclareOption{ukrainian}{%
  \def\eqdeclaration#1{, \cyrd\cyri\cyrv.\nobreakspace(#1)}%
  \def\pagedeclaration#1{, \cyrs\cyrt\cyro\cyrr.\nobreakspace#1}%
  \def\nomname{\CYRP\cyre\cyrr\cyre\cyrl\cyrii\cyrk%
         \ \cyrp\cyro\cyrz\cyrn\cyra\cyrch\cyre\cyrn\cyrsftsn}}
%    \end{macrocode}
% \end{macro}\end{macro}\end{macro}
% Finally set the default options and process everything.
%    \begin{macrocode}
\ExecuteOptions{noncompatible,notintoc,norefeq,norefpage,prefix,cfg,english}
\ProcessOptions\relax
%    \end{macrocode}
% \begin{macro}{\@outputfileextension}
% \begin{macro}{\@inputfileextension}
% The default file extension for the output and input nomenclature files are
% |.nlo| and |.nls| respectively. In compatibility mode, these are changes to |.glo|
% and |.gls|.
%    \begin{macrocode}
\if@compatibilitymode%
    \def\@outputfileextension{.glo}%
    \def\@inputfileextension{.gls}%
\else%
    \def\@outputfileextension{.nlo}%
    \def\@inputfileextension{.nls}%
\fi%
%    \end{macrocode}
% \end{macro}
% \end{macro}
% \begin{macro}{\makenomenclature}
% The definition of \cs{makenomenclature} is pretty much the same as in the \LaTeX\
% kernel for \cs{makeglossary}, we only use \cs{@nomenclature} instead of \cs{glossary}.
%    \begin{macrocode}
\def\makenomenclature{%
  \newwrite\@nomenclaturefile
  \immediate\openout\@nomenclaturefile=\jobname\@outputfileextension
  \def\@nomenclature{%
    \@bsphack
    \begingroup
    \@sanitize
    \@ifnextchar[%
      {\@@@nomenclature}{\@@@nomenclature[\nomprefix]}}%
  \typeout{Writing nomenclature file \jobname\@outputfileextension}%
  \let\makenomenclature\@empty}
%    \end{macrocode}
% \end{macro}
% \begin{macro}{\makeglossary}
% The |\makeglossary| command has been depreciated, and is only available in compatibility mode.
%    \begin{macrocode}
\if@compatibilitymode\let\makeglossary\makenomenclature\fi%
%    \end{macrocode}
% \end{macro}
% \begin{macro}{\nom@verb}
% \changes{v2.7 (1999/05/14)}{1999/05/11}{Added}
% The macro \cs{nom@verb}, which is copied from~\cite{jone:anew:1995}
% and~\cite[p.~382]{knut:thet:1984}, makes it possible to use
% \cs{nomenclature} in another macro.
%    \begin{macrocode}
\def\nom@verb{\expandafter\strip@prefix\meaning}
%    \end{macrocode}
% \end{macro}
% \begin{macro}{\nomenclature}
% \changes{v2.7 (1999/05/14)}{1999/05/11}{Protected}
% This macro just protects the ``real'' \cs{@nomenclature} macro. I am not
% sure whether this makes sense because you shouldn't use \cs{nomenclature}
% in something like \cs{section} anyway, but it doesn't hurt.
%    \begin{macrocode}
\def\nomenclature{\protect\@nomenclature}
%    \end{macrocode}
% \end{macro}
% \begin{macro}{\@nomenclature}
% \begin{macro}{\@@nomenclature}
% Without an executed \cs{makenomenclature}, \cs{@nomenclature} will only
% change some catcodes and call the macro \cs{@@nomenclature}
% to gobble its arguments.
%    \begin{macrocode}
\def\@nomenclature{%
  \@bsphack
  \begingroup
  \@sanitize
  \@ifnextchar[%
    {\@@nomenclature}{\@@nomenclature[\nomprefix]}}
\def\@@nomenclature[#1]#2#3{\endgroup\@esphack}
%    \end{macrocode}
% \end{macro} \end{macro}
% \begin{macro}{\@@@nomenclature}
% \changes{v2.7 (1999/05/14)}{1999/05/11}{More robust by using \cs{nom@verb}}
% If \cs{makenomenclature} was already executed, then \cs{@nomenclature}
% calls the macro \cs{@@@nomenclature} which writes to the nomenclature file.
% It puts the prefix in front of the entry, adds brackets |[]| around the
% entry (because it will be the argument of an \cs{item}) and adds
% possible references at the end of the entry description. A group
% is started to keep changes to the reference switches local.
% The arguments are written using \cs{nom@verb} so they will not be
% expanded, even when \cs{nomenclature} is used within another macro.
% By the way, \cs{@bsphack} and \cs{@esphack} makes \cs{nomenclature}
% disappear between two spaces; unfortunately this doesn't work if
% \cs{nomenclature} is the first thing in a line.
%    \begin{macrocode}
\def\@@@nomenclature[#1]#2#3{%
 \def\@tempa{#2}\def\@tempb{#3}%
 \protected@write\@nomenclaturefile{}%
  {\string\nomenclatureentry{#1\nom@verb\@tempa @[{\nom@verb\@tempa}]%
      \begingroup\nom@verb\@tempb\protect\nomeqref{\theequation}%
        |nompageref}{\thepage}}%
 \endgroup
 \@esphack}
%    \end{macrocode}
% \end{macro}
% \begin{macro}{\nomgroup}
% The next macro is executed between each character group in the
% nomenclature. By default it just gobbles its argument, but
% the user can redefine it to add white space or some fancy divider
% including the starting character of the new group.
%    \begin{macrocode}
\def\nomgroup#1{}
%    \end{macrocode}
% \end{macro}
% \begin{macro}{\nomlabelwidth}
% This is the default label width for the nomenclature. It can be changed
% e.\,g.~in the |cfg| file.
%    \begin{macrocode}
\newdimen\nomlabelwidth
\nomlabelwidth1cm\relax
%    \end{macrocode}
% \end{macro}
% \begin{macro}{\nom@tempdim}
% \changes{v2.9 (1999/11/23)}{1999/11/23}{New temporary dimension}
% \begin{macro}{\printnomenclature}
% \begin{macro}{\@printnomenclature}
% The optional argument is read and assigned to \cs{nom@tempdim}. Then
% the |gls| file is read.
%    \begin{macrocode}
\newdimen\nom@tempdim
\def\printnomenclature{%
  \@ifnextchar[%
    {\@printnomenclature}{\@printnomenclature[\nomlabelwidth]}}
\def\@printnomenclature[#1]{%
  \nom@tempdim#1\relax
  \@input@{\jobname\@inputfileextension}}
%    \end{macrocode}
% \end{macro} \end{macro} \end{macro}
% \begin{macro}{\printglossary}
% The |\printglossary| command has been depreciated, and is only available in compatibility mode.
%    \begin{macrocode}
\if@compatibilitymode\let\printglossary\printnomenclature\fi%
%    \end{macrocode}
% \end{macro}
% \begin{macro}{\nomlabel}
% \begin{macro}{\nompreamble}
% \begin{macro}{\nompostamble}
% \begin{macro}{\nomentryend}
% Now some bells and whistles to format the nomenclature:
% the definition of the label, the preamble, the postamble and the
% symbol that is added at the end of an entry. The last three are
% defined to do nothing by default.
%    \begin{macrocode}
\def\nomlabel#1{#1\hfil}
\def\nompreamble{}
\def\nompostamble{}
\def\nomentryend{}
%    \end{macrocode}
% \end{macro} \end{macro} \end{macro} \end{macro}
% \begin{macro}{\nomitemsep}
% \changes{v2.8 (1999/09/09)}{1999/09/09}{New skip \cs{nomitemsep}}
% The skip between two items is adjustable by changing \cs{nomitemsep}.
% It defaults to \cs{itemsep}.
%    \begin{macrocode}
\newskip\nomitemsep
\nomitemsep\itemsep
%    \end{macrocode}
% \end{macro}
% \begin{environment}{thenomenclature}
% The |thenomenclature| environment formats its title and optionally
% inserts an item in the TOC, both are dependant on
% whether the \cs{chapter} command is available or not. After
% printing the preamble, a list is started with the \cs{labelwidth}
% being set to the value defined in the optional argument of
% \cs{printnomenclature}.
%    \begin{macrocode}
\def\thenomenclature{%
  \@ifundefined{chapter}%
  {
    \section*{\nomname}
    \if@intoc\addcontentsline{toc}{section}{\nomname}\fi%
  }%
  {
    \chapter*{\nomname}
    \if@intoc\addcontentsline{toc}{chapter}{\nomname}\fi%
  }%

  \nompreamble
  \list{}{%
    \labelwidth\nom@tempdim
    \leftmargin\labelwidth
    \advance\leftmargin\labelsep
    \itemsep\nomitemsep
    \let\makelabel\nomlabel}}
\def\endthenomenclature{%
  \endlist
  \nompostamble}
%    \end{macrocode}
% \end{environment}
% \begin{macro}{\nomrefeq}
% \begin{macro}{\refpage}
% \begin{macro}{\refeqpage}
% \begin{macro}{\norefeq}
% \begin{macro}{\norefpage}
% \begin{macro}{\norefeqpage}
% These are the switches to turn referencing on or off locally for a
% single entry.
%    \begin{macrocode}
\def\nomrefeq{\@printeqreftrue}
\def\nomrefpage{\@printpagereftrue}
\def\nomrefeqpage{\@printeqreftrue\@printpagereftrue}
\def\nomnorefeq{\@printeqreffalse}
\def\nomnorefpage{\@printpagereffalse}
\def\nomnorefeqpage{\@printeqreffalse\@printpagereffalse}
%    \end{macrocode}
% \end{macro} \end{macro} \end{macro} \end{macro} \end{macro} \end{macro}
% \begin{macro}{\nomeqref}
% The equation is only referenced if the corresponding switch is
% true. Since \MakeIndex\ tends to insert a line break just before
% the page number, we have to add \cs{ignorespaces} at the end.
%    \begin{macrocode}
\def\nomeqref#1{\if@printeqref\eqdeclaration{#1}\fi\ignorespaces}
%    \end{macrocode}
% \end{macro}
% \begin{macro}{\nompageref}
% The page is also only referenced if requested. Then the end symbol is
% added and finally the group started in \cs{@@@nomenclature} is closed.
%    \begin{macrocode}
\def\nompageref#1{\if@printpageref\pagedeclaration{#1}\fi%
  \nomentryend\endgroup}
%    \end{macrocode}
% \end{macro}
% Read the config file if it exists and the corresponding option was given.
%    \begin{macrocode}
\if@loadcfg
  \InputIfFileExists{nomencl.cfg}{%
    \typeout{Using the configuration file nomencl.cfg}}{}
\fi
%    \end{macrocode}
% The end.
%    \begin{macrocode}
%</package>
%    \end{macrocode}
%
%
% \subsection{The \MakeIndex\ Style File}
% \label{sec:ist}
%
% The ``magic word'' for \MakeIndex\ in the input file is \cs{nomenclatureentry}.
% German user might need to redefine the quote character if they want to use
% |"a| instead of |\"a|. Choose whatever character you see fit except \verb+|+,
% |@| and |!|.
%    \begin{macrocode}
%<*idxstyle>
%% ---- for input file ----
keyword    "\\nomenclatureentry"
%% Germans might want to change this and delete the two %%
%% quote '"'
%    \end{macrocode}
% Define what is printed at the beginning and the end of the file and
% the skip between groups. Since we already write \cs{nomgroup} between
% groups, we define |group_skip| to just input an empty line.
%    \begin{macrocode}
%% ---- for output file ----
preamble   "\\begin{thenomenclature} \n"%
postamble  "\n\n\\end{thenomenclature}\n" group_skip "\n"
%    \end{macrocode}
% Since we can't handle multiple pages for an entry anyway, we also
% don't need any delimiters.
%    \begin{macrocode}
delim_0    ""
delim_1    ""
delim_2    ""
%    \end{macrocode}
% Now the macro between the groups. Since the flag is positive, the
% character will be inserted as a capital letter. As the comment
% states, this will cause some warnings. If someone has a better
% solution, let me know.
%    \begin{macrocode}
%% The next lines will produce some warnings when
%% running Makeindex as they try to cover two different
%% versions of the program:
lethead_prefix "\n \\nomgroup{"
lethead_suffix "}\n"
lethead_flag   1
heading_prefix "\n \\nomgroup{"
heading_suffix "}\n"
headings_flag  1
%</idxstyle>
%    \end{macrocode}
%
% \Finale
